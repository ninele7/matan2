\section{Понятие функции многих переменных, скалярной и векторной. Определение предела скалярной функции многих переменных по Коши и по Гейне, их эквивалентность. Предел и арифметические операции.}

Пусть $X \subseteq \mathbb{R}^n$, $Y \subseteq \mathbb{R}$.

Если каждому элементу $x \in X$ по определенному правилу поставлено в соответствие единственное число $y \in Y$,
то говорят, что на множестве $X$ задана \textbf{функция многих
переменных}. Если это правило обозначить $f$, то $X$ называют \textit{областью определения}, 
а $Y$  \textit{областью значений} функции $f$. Обозначения:

$$y=f(x); x \in X \subseteq \mathbb{R}^n,$$

т. е. $x = (x_1; x_2; \ldots ; x_n)$, $x_i \in \mathbb{R},$ или

$$y = f(x_1; x_2; \ldots ; x_n); (x_1; x_2; \ldots ; x_n) \in X.$$

Если $n=1$, т.е. значением отображения является действительное число (\textit{скалярная величина}), отображение называют
\textbf{скалярной} функцией нескольких переменных. Если же $n>1$, то указанное отображение
называют \textbf{векторной} функцией нескольких переменных (или векторной функцией
векторного аргумента).

\;

\textbf{Определение} (определение предела по Коши).

Пусть $a$ предельная точка множества $X$ ($a \in X'$). Число
$A \in \mathbb{R}$ называется пределом функции $f(x)$ в точке $a$, если

$$\forall O(A) \exists O(a) \forall x \in X \cap O (a) \Rightarrow f(x) \in O(A).$$

Обозначение: $A = \underset{x \rightarrow a} {lim} f(x)$.

Если указаны радиусы окрестностей, то это определение запишется в виде

$$\forall \epsilon > 0 \;\; \exists \delta = \delta(\epsilon) > 0 \;\; \forall x \in X \cap O_{\delta}(a) \Rightarrow f(x) \in O_{\epsilon}(A),$$

или

$$\forall \epsilon > 0 \;\; \exists \delta = \delta (\epsilon) > 0 \;\; \forall x \in X \;\;\;
(0 < \rho (x,a) < \delta \Rightarrow \vert f(x) - A \vert < \epsilon),$$

или 

$$\forall\epsilon>0\;\;\exists\delta=\delta(\epsilon)>0\;\;\forall{x}\in{X}$$

$$
\begin{pmatrix}
0<\vert{x_1-a_1}\vert<\delta, & \; \\
0<\vert{x_2-a_2}\vert<\delta, & \; \\
\vdots & \Rightarrow\vert{f(x)-A}\vert<\epsilon  \\
0<\vert{x_n-a_n}\vert<\delta, & \;
\end{pmatrix}
.$$

\;

\textbf{Определение} (определение предела по Гейне).

Пусть $a\in{X'}$. Число $A\in{\mathbb{R}}$ называется пределом функции $f$ в точке $a$, если

$$\forall\lbrace{x^p}\rbrace\subseteq{X},\;\;x^p\neq{a},\;\;x^p\rightarrow{a}\Rightarrow{f(x^p)\rightarrow{A}}$$

\;

\textbf{Теорема} (об эквивалентности определений предела). 

Пусть $y = f(x)$; $x\in{X}\subseteq\mathbb{R}^n$, $a\in{X'}$ и $A\in\mathbb{R}$.

Тогда

$A=\underset{x\rightarrow{a}}{lim}f(x)$\textit{(по Гейне)} $\Leftrightarrow\;\;A=\underset{x\rightarrow{a}}{lim}f(x)$\textit{(по Коши)}.

\textit{Доказательство} легко проводится по той же схеме, что и доказательство этой теоремы для случая функции одной переменной.

\;

На множестве $Y$ всех функций вида $f:X\subseteq\mathbb{R}^n\rightarrow{R}$, как и в случае скалярных функций, 
можно ввести операции сложения функций и умножения функций на действительные числа.

\textbf{Суммой} функций нескольких переменных $f,g\in{Y}$ называют такую функцию $f+g\in{Y}$, 
что для любого $x\in{X}$ верно равенство $(f+g)(x)=f(x)+g(x)$, в правой части которого стоит сумма значений векторных функций, 
являющихся элементами линейного пространства $\mathbb{R}$. 

Аналогично \textbf{произведением} функции нескольких переменных $f\in{Y}$ на действительное число $\lambda$ 
называют такую функцию $(\lambda{f})\in{Y}$, 
что для любого $x\in{X}$ верно равенство $(\lambda{f})(x)=\lambda{f}(x)$, в правой части которого стоит произведение вектора 
$f(x)\in\mathbb{R}$ на действительное число $\lambda$.