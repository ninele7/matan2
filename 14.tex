\section{Вторая теорема Больцано–Коши о промежуточном значении непрерывной функции.}
    
Теорема
    
Пусть $G$ - связное множество в $R^n$. Путь функция $f$ непрерывна на $G$ и $\exists a \in G$ и $\exists b \in G$ такие, что $f(a) \not= f(b)$. Тогда для любого числа $C$, заключенного между $f(a)$ и $f(b)$, существует точка $c \in G$ такая, что $f(c) = C$.
    
Доказательство
    
Так как $G$ - связное множество, $\exists$ непрерывная кривая L
    
\begin{displaymath}
    L: 
    \begin{cases} 
        x_1 = \phi_1(t), \\ 
        x_1 = \phi_2(t), \\
        ... \\
        x_n = \phi_n(t), \\
        \alpha \leq t \leq \beta
    \end{cases}
\end{displaymath}
    
соединяющая точки $a$ и $b$ и лежащая в $G$, т.е. $a = (\phi_1(\alpha), \phi_2(\alpha), ..., \phi_n(\alpha)), b = (\phi_1(\beta), \phi_2(\beta), ..., \phi_n(\beta))$ и $x = (\phi_1(t), \phi_2(t), ..., \phi_n(t)) \in G$ при любом $t\in [\alpha, \beta]$. 
    
Пусть $F(t) = f(\phi_1(t), \phi_2(t), ..., \phi_n(t))$. По теорема о непрерывности сложной функции $F(t)$ непрерывна на $[\alpha, \beta]$ и $F(\alpha) = f(a), F(\beta) = f(b)$, т.е. $F(a) \not= F(b)$ и $C$ между $F(\alpha)$ и $F(\beta)$. Тогда по теореме о промежуточном значении непрерывной функции одного переменного найдется такая точка $\gamma \in (\alpha, \beta)$, что $F(\gamma) = C$, а тогда $c = (\phi_1(\gamma), \phi_2(\gamma), ..., \phi_n(\gamma)) \in G$ - искомая точка.