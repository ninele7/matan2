\section{Определение частной производной, дифференцируемости скалярной функции многих переменных.
Эквивалентность двух определений дифференцируемости.}

\subsection{Определние частной производной}
$$
f: D\in \R^m \rightarrow \R
$$
Пусть функция $f(x_0, y_0)$ определена в некоторой окрестности точки $(x_0, y_0)$. Если существуют конечные пределы 
\begin{gather*}
\lim_{\Delta x \rightarrow 0}{\frac{f(x_0 + \Delta x, y_0) - f(x_0, y_0)}{\Delta x}}
\\
\lim_{\Delta y \rightarrow 0}{\frac{f(x_0, y_0 + \Delta y) - f(x_0, y_0)}{\Delta y}}
\end{gather*}
то эти пределы есть частные производные функции f в точке $(x_0, y_0)$ соответственно и обозначают $\frac{df(x_0, y_0)}{dx}  \frac{df(x_0, y_0)}{dy}$
Если говорить совсем простым языком, то частная производная по какой-либо переменной - это производная, где все остальные переменные (кроме той, конечно, по которой мы берем производную) принимаются в качестве константы.
\\
С функциями более двух переменных все происходит абсолютно аналогично: это будет производная, обозначающаяся через $\frac{df(x_1,...,x_k,...,x_n)}{dx_k}$
Для вычисления такой производной пользуются всеми стандартными правилами, и, как было сказано выше, просто принимают остальные переменные за константы.

\subsection{Определения дифференцируемости скалярной функции многих переменных и их эквивалентность}
Приращением функции f в точке M0 называется функция $\Delta f(M_0) = f(M) - f(M_0)$
Рассмотрим произвольное приращение функции в точке M0. 
Дифференцируемость возьмем аналогично определению дифференцируемости функции одной переменной. Напомню:
$$
\Delta f(x_0) = A \Delta x + o(\Delta x)
$$
Аналогично для нашего случая:
$$
\Delta f(M_0) = f(M) - f(M_0) = A_1\Delta x_1 + ... + A_m\Delta x_m + \varepsilon_1 \Delta x_1 + ... + \varepsilon_m \Delta x_m
$$
Где 
$$
A_i \in \R, \varepsilon_i = \varepsilon_i(\Delta x_1, ..., \Delta x_m) \rightarrow 0
$$
При $\Delta x_1 \rightarrow 0 ... \Delta x_m \rightarrow 0$
\\
В силу критерия сходимости в n-мерном пространстве, это эквивалентно тому, что если мы берем $\Delta x = (\Delta x_1, ..., \Delta x_m)$ - вектор приращения, то $\Delta x \rightarrow 0$ в $\R^m$
То есть вектор приращения бесконечно малый.
\\
\\
Закрепим определение:
Функция является дифференцируемой в точке М0, если найдутся такие константы А1, ..., Аm и функции $\varepsilon_1 ... \varepsilon_m$, зависящие от вектора приращения $\Delta x$ и являющиеся бесконечно малыми при $\Delta x \rightarrow 0$, что приращение функции представимо в показанном выше виде.
\\
Эквивалентное определение:
Функция дифференцирема в точке М0, если ее приращение в этой точке, соответствующее вектору $\Delta x$ представимо в виде:
$$
\Delta f(M_0) = f(M) - f(M_0) = \Sigma_{i = 1 .. m} A_i \Delta x_i + o(\rho)
$$
Здесь $\rho$ - расстояние между точками М  и М0, что в нашем контексте - $\Delta x$ - вектор приращения.
$$
\rho = |\Delta x| = \sqrt{\Sigma{\Delta x_i^2}}
$$
Докажем эквивалентность этих определений
Необходимо доказать, что
$$
\varepsilon_1 \Delta x_1 + ... + \varepsilon_m \Delta x_m = o(\rho)
$$
Докажем слева направо. Это будет верно в том случае, если:
\begin{gather*}
\frac{\varepsilon_1 \Delta x_1 + ... + \varepsilon_m \Delta x_m}{\rho} \rightarrow 0 (\rho \rightarrow 0)
\\
\frac{\varepsilon_1 \Delta x_1}{\rho} + ... + \frac{\varepsilon_m \Delta x_m}{\rho}
\\
\mid\frac{\Delta x_i}{\rho}\mid = \frac{\mid\Delta x_i \mid}{\sqrt{\Delta x_1^2 + ... + \Delta x_m^2}} < 1
\end{gather*}
Значит они ограниченные. Что мы можем сказать про функции $\varepsilon_i$? Они бесконечно малые при $\Delta x \rightarrow 0$, а произведение б.м. на ограниченную есть бесконечно малая. Доказано. Теперь в обратную сторону.
\\
\begin{gather*}
    o(\rho) = \frac{o(\rho) * \rho^2}{\rho^2} = \frac{o(\rho)}{\rho} * \frac{\Delta x_1^2 + ... + \Delta x_m^2}{\rho} = \frac{o(\rho)}{\rho} * \frac{\Delta x_1}{\rho} * \Delta x_1 + ... + \frac{o(\rho)}{\rho} * \frac{\Delta x_m}{\rho} * \Delta x_m
\end{gather*}
Возьмем для каждого такого слагаемого $\varepsilon_i = \frac{o(\rho)}{\rho} * \frac{\Delta x_i}{\rho}$ и получим эквивалентной определение. Доказано.
