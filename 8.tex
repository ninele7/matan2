
\section{Понятие предела векторной функции многих переменных.}


\textbf{Теорема} 

Пусть функция $f(x;y)$ определена на множестве $X\times{Y}$ и $x_0\in{X'}, y_0\in{Y'}$. 
Если существует $\underset{y\rightarrow{y_0}}{\underset{x\rightarrow{x_0}}{lim}}f(x,y)=A$ и для любого $x\in{X}$, $x\neq{x_0}$, 
существует $\underset{y\rightarrow{y_0}}{lim}f(x,y)=\varphi(x)$, то существует и 
$\underset{x\rightarrow{x_0}}{lim}\underset{y\rightarrow{y_0}}{lim}f(x,y)$, он равен $A$.

\textit{Доказательство.} По определению двойного предела

$$\forall\epsilon>0\;\exists\delta=\delta(\epsilon)>0\;\forall(x,y)\in{X}\times{Y}$$
$$
\begin{pmatrix}
0<\vert{x-x_0}\vert<\delta, & \; \\
\; & \Rightarrow\vert{f(x,y)-A}\vert<\frac{\epsilon}{2} \\
0<\vert{y-y_0}\vert<\delta, & \;
\end{pmatrix}
.$$

По определению однократного предела для любого $x\in{X}$, $x\neq{x_0}$,

$$\forall\epsilon>0\;\;\exists\tilde{\delta}=\tilde{\delta}(\epsilon)>0\;\;\forall{y}\in{Y}$$
$$(0<\vert{y-y_0}\vert<\tilde{\delta}\Rightarrow\vert{f(x,y)-\varphi(x)}\vert<\frac{\epsilon}{2})$$

Возьмем $x\in{X}$ из проколотой $\delta$-окрестности точки $x_0$ и рассмотрим разность $\varphi(x)-A$. 
Прибавим и отнимем в этом выражении $f(x; y)$ с $y\in@Y$, $0<\vert{y-y_0}\vert<min\lbrace\delta,\tilde{\delta}\rbrace$, 
и получим оценку

$$\vert\varphi(x)-A\vert=\vert\varphi(x)\pm{f(x,y)}-A\leq\vert\varphi(x)-f(x,y)\vert+\vert{f(x,y)-A}\vert<\frac{\epsilon}{2}+\frac{\epsilon}{2}=\epsilon,$$

т. е. $\underset{x\rightarrow{x_0}}{lim}\underset{y\rightarrow{y_0}}{lim}f(x,y)=A$.
