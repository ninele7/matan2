\section{Понятие предела векторной функции многих переменных.}

\begin{definition}
    Вектор $a$ называется пределом векторной функции $r(t),t\in{X}$, 
    при $t\to{t_0}$(или в точке $t\to{t_0}$), если

    $$\lim_{t\to{t_0}}\vert{r(t)-a}\vert=0\eqno(1)$$

    В этом случе пишут

    $$\lim_{t\to{t_0}}r(t)=a\eqno(2)$$
\end{definition}

В этом определении $\vert{r(t)-a}\vert$ -- числовая функция. Таким образом,
понятие предела векторной функции сводится к понятию предела скалярной функции (1).
Вспомнив определение этого понятия, получим, что (2) означает 

$$\forall\varepsilon\exists\delta:\;\forall{t}\in{X}\cap{U}(t_0;\delta)\eqno(3)$$

выполняется неравенство

$$\vert{r(t)-a}\vert<\varepsilon\eqno(4)$$

Как и в случае скалярных функций, будем полагать, что $t_0$ -- точка прикосновения (конечная
или бесконечно удалённая) множества $X$. Если $t_0$ -- конечная точка, то условие (3) можно записать 
в виде

$$\vert{t-t_0}\vert<\delta,\;t\in{X},$$

а если $t_0$ -- одна из бесконечно удалённых точек $\infty,+\infty$ или $-\infty$, то соответственно 
в одном из следующих трёх видов:

$$\vert{t}\vert>\frac{1}{\delta},t>\frac{1}{\delta} or t<-\frac{1}{\delta},$$

где всюду $t\in{X}$.

$\;$

\textit{Свойства пределов векторных функций}

\begin{enumerate}
    \item Если $\lim_{t\to{t}}r(t)=a$, то $\lim_{t\to{t_0}}\vert{r(t)}\vert=\vert{a}\vert$
    Это непосредственно следует из неравенства $\vert\vert{r}\vert-\vert{a}\vert\vert\leq\vert{r-a}\vert$.
    Геометрический смысл этого неравенства состоит в том, что разность длин двух сторон треугольника 
    не превышает длины его третьей стороны.
    \item $\lim_{t\to{t}}[r_1(t)+r_2(t)]=\lim_{t\to{t_0}}r_1(t)+\lim_{t\to{t_0}}r_2(t)$.
    \item $\lim_{t\to{t_0}}f(t)r(t)=\lim_{t\to{t_0}}f(t)\lim_{t\to{t_0}}r(t)$ ($f(t)$ -- скалярная функция).
    \item $\lim_{t\to{t_0}}r_1(t)r_2(t)=\lim_{t\to{t_0}}r_1(t)\lim_{t\to{t_0}}r_2(t)$.
    \item $\lim_{t\to{t_0}}r_1(t)\times{r_2(t)}=\lim_{t\to{t_0}}r_1(t)\times\lim_{t\to{t_0}}r_2(t)$.
\end{enumerate}
В свойствах 1-5 все рассматриваемые функции определены на некотором множестве $X\subset\mathbb{R}$. 
В свойствах 2-5 предполагается, что все пределы, входящие в правые части равенств, существуют и утверждается,
что существуют пределы, стоящие в левых частях равенств, причём имеют место написанные формулы.