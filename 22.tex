\section{Инвариантность формы первого дифференциала}

Пусть функция $ u = f(x,y,z) $ имеет непрерывные частные~производные
$ {u'}_{x} $ , $ {u'}_{y} $ , $ {u'}_{z}\  $причем х, у, z, в свою
очередь, являются функциями от новых переменных t и $ v $:

$$ x = \ \phi\left( t,v \right)\ ,\ \ y = \psi\left( t,v \right),\ \ z = X\left( t,v \right), $$
также имеющими непрерывные же частные производные $ x_{t}x_{v}$, $y_{t}y_{v}$, $z_{t}z_{v} $. Тогда не только существуют производные от
сложной~функции и по t но эти производные также непрерывны по t

Если бы х, у и z были независимыми переменными, то, как мы знаем, полный
дифференциал~функции и был бы равен:

$$ du = \ {u'}_{x}dx + {u^{'}}_{y}dy + \ {u^{'}}_{z}\text{dz} $$

В данном же случае и зависит - через посредство х, у, z - от переменных
t и $ v $. Следовательно, по отношению к этим переменным, дифференциал
напишется так:

$$ du = \ {u'}_{t} \cdot dt + {u^{'}}_{v} \cdot \text{dv} $$

$$ {\text{\ \ u}^{'}}_{t} = {u^{'}}_{x} \cdot {x^{'}}_{t} + {u^{'}}_{y} \cdot \ {y'}_{t} + {u^{'}}_{z} \cdot {z'}_{t} $$

Аналогично:

$$ {\text{\ \ u}^{'}}_{v} = {u^{'}}_{x} \cdot {x^{'}}_{v} + {u^{'}}_{y} \cdot \ {y'}_{v} + {u^{'}}_{z} \cdot {z'}_{v} $$

Подставив эти значения в выражение для$ \ \text{du} $, будем иметь:

$$ du = \left( {u^{'}}_{x}{u^{'}}_{t} + {u^{'}}_{y}{u^{'}}_{t} + {u^{'}}_{z}{u^{'}}_{t} \right)dt + \left( {u^{'}}_{x}{u^{'}}_{v} + {u^{'}}_{y}{u^{'}}_{v} + {u^{'}}_{z}{u^{'}}_{v} \right)\text{dv} $$

Перегруппируем члены следующим образом:

$$ du = \ u_{x}\left( {x^{'}}_{t}dt + {x^{'}}_{v}\text{dv} \right) + {u^{'}}_{y}\left( {y^{'}}_{t}dt + {y^{'}}_{v}\text{dv} \right) + {u^{'}}_{z}({z^{'}}_{t}dt + {z^{'}}_{v}dv) $$

Нетрудно видеть, что выражения, стоящие в скобках, суть не что иное, как
дифференциалы~функций х, у, z (от $ u $ и $ v)\  $так что:

$$ du = \ u_{x}dx + u_{y}\text{dy} + u_{z}\text{dz} $$

\textbf{Мы пришли к той же самой форме дифференциала, что и в случае,
когда х, у, z были независимыми переменными}

Итак, \textbf{для функций~нескольких переменных имеет место
инвариантность формы первого дифференциала, как и для функций одной
переменной}

Может случиться, что х, у и z будут зависеть от различных переменных,
например:

$$ x = \ \phi\left( t \right),\ y = \Psi\left( t \right),z = \chi(t)\  $$

В таком случае мы всегда можем считать, что:

$$ x = \phi_{1}\left( t,v,w \right),\ {y = \ \varphi}_{1}\left( t,v,w \right),{z = \ \chi}_{1}\left( t,v,w \right), $$

и все предыдущие рассуждения будут применимы и к этому случаю.

Следствия. Для случая, когда х и у были функциями одной переменной, мы
имели следующие формулы:

$ d\left( \text{cx} \right) = c \cdot \text{dx} $\emph{,}
$ d\left( x \pm y \right) = dx \pm dy $\emph{, d(xy) =
y}$ \cdot dx + x \cdot dy $\emph{,}

$$ d\left( \frac{x}{y} \right) = \frac{y \cdot dx + x \cdot dy}{y^{2}} $$

Эти формулы верны и в том случае, когда х и у являются функциями любого
числа переменных, т. е. когда:

$ x = \varphi(t,v,\ \ldots) $, $ y = \psi(t,v,\ldots) $

Докажем последнюю формулу.

Для этого примем сначала х и у за независимые переменные; тогда

$$ d\left( \frac{x}{y} \right) = \frac{1}{y}\  \cdot dx - \ \frac{x}{y^{2}}\  \cdot dy = \frac{y \cdot dx - x \cdot dy}{y^{2}} $$

Видим, что при этом предположении дифференциал имеет тот же вид, что и
для функций $ x $ и $ \text{y\ } $одной переменной. На основании же
инвариантности формы дифференциала можно утверждать, что эта формула
справедлива и в том случае, когда $ x $ и $ y $ являются функциями
любого числа переменных.
