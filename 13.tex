\section{Понятие сложной функции. Теорема о непрерывности сложной функции.}

Теорема о непрерывности сложной функции. 
Пусть отображение $\phi : \R^m \rightarrow \R^n$ определено в некоторой окрестности точки $t_0 = (t^0_1, t^0_2, ..., t^0_m) \in R^m$ и непрерывным образом отображает ее в точку $x_0 = (x_1^0, x_2^0, ..., x_n^0) \in R^n$. Пусть функция $f: \R^n \rightarrow \R^m$ определена в некоторой окрестности точки $x_0$ и непрерывна в этой точке. Тогда сложная функция $F(t) = f(\phi (t)), F:\R^m \rightarrow \R^n$ непрерывна в точке $t_0$.

\subsection{Доказательство}

Заметим, что для отображения 

\begin{displaymath}
    \phi (t) = (\phi _1(t), \phi _2(t), ..., \phi _n(t))
\end{displaymath}

непрерывность в точке $t_0$ означает непрерывность каждой из функций $\phi _i(t)$ в точке $t_0$ как функции m переменных, т.е. если $\{t^p\} \subset R^m$ и $t^p \underset{p \rightarrow \infty}{\longrightarrow} t_0$, то

\begin{displaymath}
    \phi (t^p) = (\phi _1(t^p), \phi _2(t^p), ..., \phi _n(t^p)) \underset{p \rightarrow \infty}{\longrightarrow} 
\end{displaymath}

\begin{displaymath}
    \underset{p \rightarrow \infty}{\longrightarrow} (\phi _1(t_0), \phi _2(t_0), ..., \phi _n(t_0)) = (x_1^0, x_2^0, ..., x_n^0) = x_0.
\end{displaymath}

Положим $x^p = \phi (t^p)$, тогда $x^p \underset{p \rightarrow \infty}{\longrightarrow} x_0$. В силу непрерывности функции $f$, $f(x^p) \underset{p \rightarrow \infty}{\longrightarrow} f(x_0)$, т.е. 

\begin{displaymath}
    F(t^p) = f(\phi (t^p)) \underset{p \rightarrow \infty}{ \longrightarrow} f(x_0) = f(\phi (t_0)) = F(t_0)
\end{displaymath}
