\section{Понятие производной скалярной функции по направлению. Вектор градиент, его свойства.}

Рассмотрим функцию $u(x, y, z)$ в точке $М(x,y,z)$ и точке $М_1(x+D_x,y+D_y,z+D_z)$. 
Проведем через точки $M$ и $M_1$ вектор. Углы наклона этого вектора к направлению координатных осей $x$, $y$, $z$ обозначим соответственно $\alpha$, $\beta$, $\gamma$. 
Косинусы этих углов называются \textit{направляющими косинусами} вектора.
Расстояние между точками $М$ и $М_1$ на векторе обозначим $DS$.

$$\Delta{S}=\sqrt{\Delta{x^2}+\Delta{y^2}+\Delta{z^2}}$$

Далее предположим, что функция $u(x,y,z)$ непрерывна и имеет непрерывные частные производные по переменным $x$, $y$ и $z$. 
Тогда правомерно записать следующее выражение:

$$\triangle{u}=\frac{\delta{u}}{\delta{x}}\Delta{x}+\frac{\delta{u}}{\delta{y}}\Delta{y}+\frac{\delta{u}}{\delta{z}}\Delta{z}+\varepsilon_1\Delta{x}+\varepsilon_2\Delta{y}+\varepsilon_3\Delta{z}$$

где величины $\varepsilon_1,\varepsilon_2,\varepsilon_3$ - бесконечно малые при $\Delta{S}\rightarrow{0}$.

Из геометрических соображений очевидно:

$$\frac{\Delta{x}}{\Delta{S}}=\cos{\alpha}\;\;\;\;\frac{\Delta{y}}{\Delta{S}}=\cos{\beta}\;\;\;\;\frac{\Delta{z}}{\Delta{S}}=\cos{\gamma}$$

Таким образом, приведенные выше равенства могут быть представлены следующим образом:

$$\frac{\Delta{u}}{\Delta{S}}=\frac{\delta{u}}{\delta{x}}\cos{\alpha}+\frac{\delta{u}}{\delta{y}}\cos{\beta}+\frac{\delta{u}}{\delta{z}}\cos{\gamma}+\varepsilon_1\cos{\alpha}+\varepsilon_2\cos{\beta}+\varepsilon_3\cos{\gamma}$$

$$\frac{\delta{u}}{\delta{s}}=\lim_{\Delta{S}\to{0}}\frac{\Delta{u}}{\Delta{S}}=\frac{\delta{u}}{\delta{x}}\cos{\alpha}+\frac{\delta{u}}{\delta{y}}\cos{\beta}+\frac{\delta{u}}{\delta{z}}\cos{\gamma}$$

Заметим, что величина $s$ является скалярной. Она лишь определяет направление вектора $\vec{S}$.

Из этого уравнения следует следующее определение:

\begin{definition}
    Предел $\lim_{\Delta{S}\to{0}}\frac{\Delta{u}}{\Delta{S}}$ называется \textit{производной функции $u(x,y,z)$ по направлению вектора} 
    в точке $\vec{S}$ с координатами $(x,y,z)$.
\end{definition}

\begin{definition}
    Если в некоторой области $D$ задана функция $u=u(x,y,z)$ и некоторый вектор, 
    проекции которого на координатные оси равны значениям функции $u$ в соответствующей точке $\frac{\delta{u}}{\delta{x}}$; $\frac{\delta{u}}{\delta{y}}$; $\frac{\delta{u}}{\delta{z}}$
    то этот вектор называется \textit{градиентом функции $u$}.
\end{definition}

$$grad\;u=\frac{\delta{u}}{\delta{x}}\vec{i}+\frac{\delta{u}}{\delta{y}}\vec{j}+\frac{\delta{u}}{\delta{z}}\vec{k}$$

\subsection{Свойства:}

1. Производная функции $u=u(x,y,z)$ по направлению вектора $\vec{S}$ достигает своего наибольшего значения, если направление вектора $\vec{S}$ 
совпадает с направлением градиента этой функции.

$\;$

Действительно, производную данной функции по направлению вектора $\vec{S}$ можно записать следующим образом 
$\frac{\delta{u}}{\delta{S}}=(grad\;u\cdot\vec{e_S})=\vert{grad\;u}\vert\cdot\vert\vec{e_S}\vert\cdot\cos{\phi}$, где $\phi$ - угол между градиентом и вектором $\vec{S}$. 
Если этот угол равен нулю $\phi=0$, то косинус этого угла и производная функции принимают наибольшие значения, $\cos{0}=1$, $\frac{\delta{u}}{\delta{S}}=\vert{grad\;u}\vert\cdot\vert\vec{e_S}\vert\cdot\cos{0}$.

$\;$

2. Производная функции $u=u(x,y,z)$ по направлению вектора $\vec{S}$ равняется нулю, если направление вектора $\vec{S}$ перпендикулярно направлению градиента этой функции.
Действительно, $\frac{\delta{u}}{\delta{S}}=\vert{grad\;u}\vert\cdot\vert\vec{e_S}\vert\cdot\cos{90^\circ}=0$.
Данные свойства используются при решении задач оптимизации (нахождения наибольшего, наименьшего значений функций) с помощью численных методов. 
Градиент функции определяет направление наибольшего изменения функции. 
Направление перпендикулярное градиенту определяет направление, в котором функция не изменяется.

$\;$

Известно, что на поверхности уровня $u(x,y,z)=c$ функция $u=u(x,y,z)$ не изменяется. 
Следовательно, градиент функции перпендикулярен поверхности уровня. 
Это обстоятельство можно использовать для написания уравнения касательной плоскости к поверхности $u(x,y,z)=c$. 
Пусть точка $M_0(x_0,y_0,z_0)$ принадлежит поверхности. 
Найдем градиент функции $u=u(x,y,z)$ в этой точке $grad\;u\vert{M_0(x_0,y_0,z_0)}=(\frac{\delta{u}}{\delta{x}}\vert{M_0,})$и напишем уравнение плоскости, 
проходящей через точку $M_0$ перпендикулярно вектору $\vec{grad\;u}\vert{M_0}$. Получаем уравнение касательной плоскости 
$\frac{\delta{u}}{\delta{x}}\vert{M_0}(x-x_0)+\frac{\delta{u}}{\delta{y}}\vert{M_0}(y-y_0)+\frac{\delta{u}}{\delta{z}}\vert{M_0}(z-z_0)=0$.
