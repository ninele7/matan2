\newcommand\R{\ensuremath{\mathbb{R}}}
\section{Элементарные свойства функций многих переменных, связанные с пределами.}

\subsection{Единственность предела функции}

Следует из единственности предела последовательности по Гейне 
(отсылка к 7 билету)
$$
f(x^{(n)}) \longrightarrow A
$$


\subsection{Локальная ограниченность}

Если в точке функция имеет конечный предел = А, то найдется 
такая проколотая окрестность точки а(маленькое), в которой функция 
ограничена. Можно записать как определение по Коши, но с небольшим 
дополнением:
$$
\forall \varepsilon > 0 \ \exists \delta(\varepsilon) > 0
\ : \ \forall x\in D \ \cap \ B(a,\delta(\varepsilon)), x\ne a
\Rightarrow \mid f(x) - A \mid < \varepsilon
$$
где $D \subseteq \R^m$
\\
\\
Зафиксируем здесь $\varepsilon = 1$ и получим, что 
$A-1<f(x)<A+1$


\subsection{Отделимость от нуля}

Если $\exists \lim_{x\to a}{f(x)} = A \ne 0$, то 
$$
\exists R>0 \ \exists \delta > 0 \ : \ \forall x\in D \ \cap \ 
B(a,\delta(\varepsilon)), x\ne a, \mid f(x) \mid > R
$$
Для доказательства посмотрим на опредление по Коши и заметим, что 
в силу обратного неравенства треугольника мы можем написать в 
правой части определения 
$$
\vert\mid f(x)\mid - \mid A\mid\vert \ \leq \ \mid f(x) - A\mid \ < \ \varepsilon
$$
Тогда можем заметить, что при раскрытии модуля получится следующее:
$$
\mid A\mid - \ \varepsilon \ \leq \ \mid f(x)\mid \ \leq \ \mid A\mid + \ \varepsilon
$$
Рассмотрим левую часть двойного неравенства. Так как эпсилон у нас любое, то мы имеем право взять 
$\varepsilon = {\mid A\mid\over2} > 0 \ $(т.к. $A\ne0$). Тогда получим в левой части
$\mid A\mid - {\mid A\mid\over2} = {\mid A\mid\over2}$ и возьмем его за R. Вот мы и получили
определение отделимости от нуля, где $\delta = \delta(\varepsilon = \mid A\mid)$


\subsection{Арифметические свойства}

$$
\exists \lim_{x\to a}{f(x)} = A \ \ \exists \lim_{x\to a}{g(x)} = B
$$
\subsubsection{Сложение(вычитание)}
$$
\lim_{x\to a}{(f(x) + g(x))} = A + B
$$
\subsubsection{Умножение}
$$
\lim_{x\to a}{(f(x) * g(x))} = A * B
$$
\subsubsection{Деление}
$$
B \ne0 \Rightarrow \lim_{x\to a}{f(x)\over g(x)} = {A\over B}
$$


\subsection{Неравенства}
Если две функции имеют предел и связаны каким-то неравенством, то пределы будут связаны таким же неравенством
$$
f(x) \leq g(x), x\in B(a, R), x\ne a \ \exists \lim_{x\to a}{f(x)} = A \ \exists\lim_{x\to a}{g(x)} = B \ \Rightarrow \ A\leq B
$$
В другую сторону так работает только со строгим неравенством:
$$
\exists \lim_{x\to a}{f(x)} = A \ \exists\lim_{x\to a}{g(x)} = B \ A < B \Rightarrow 
\exists \varepsilon > 0 \ : \ \forall x \in B(a, \varepsilon), x\ne a \Rightarrow f(x) < g(x)
$$
