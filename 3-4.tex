\section{Евклидово пространство, метризуемость евклидова пространства, неравенство Коши-Буняковского}
\subsection{Евклидово пространство}

Пусть $X$ - векторное просранство над полем $\mathbb {R}$. Введем отображение из $X \times X$ в $\mathbb {R}$, называемое скалярным произедением этих векторов, которое любой упорядоченной паре векторов $(x, y) \in X \times X$ ставит в соответствие число из $\mathbb {R}$, таким образом, что


\begin{enumerate} 
  \item $\forall x,y \in X \quad (x, y) = (y, x)$,
  \item  $\forall x,y,z \in X \quad (x + y, z) = (x, z) + (y, z)$,
  \item $\forall x,y \in X \quad \forall \lambda \in \mathbb {R} \quad (\lambda x, y) = \lambda(x, y)$
  \item $\forall x \in X (x, x) \geq 0$, причем $(x, x) = 0$ тогда и только тогда, когда $x = 0$.
\end{enumerate}
Линейное(векторное) пространство со скалярным произведением называется евклидовым пространством.

Любое евклидово пространство является метрическим, если $\rho(x, y) = \sqrt{(x - y, x - y)}$. Убедимся в том, что введенная метрика удовлетворяет аксиомам метрики
\begin{enumerate} 
  \item $\rho(x, y) \geq 0$, - очевидно, корень квадратный всегда $\geq 0$ 
  \item  $\rho(x, y) = 0 \Leftrightarrow x-y = 0, \ x=y$, - очевидно
  \item $\rho(x, y) = \sqrt{(x - y, x - y)} = \sqrt{-(y-x, x - y)} = \sqrt{-(x -y, y - x)} = \sqrt{(y - x, y - x)}=\rho(y, x)$, по свойствам скалярного произведения (1) и (3) 
  \item $\rho(x, y) \leq \rho(x, z) + \rho(y, z)$ (докажем это утверждение после введения неравенства Коши-Буняковского)
\end{enumerate}
\subsection{Неравенство Коши-Буняковского}
Введем определение длины вектора (модуль вектора) как расстояние от нулевого вектора до данного $\rho(x, 0) = \sqrt{(x, x)}$. Будем обозначать эту величину как модуль вектора, то есть $\rho(x, 0) = \sqrt{(x, x)} = |x|$

Теперь рассмотрим функцию 
$$
\forall x,y \in X, \ \forall t \in \mathbb {R}, \quad \Phi(t) = (x +ty, x + ty)
$$
Раскроем скобки и преобразуем, воспользовавшись свойствами скалярного произведения (1), (2), (3)
$$
\Phi(t) = (x +ty, x + ty) = (x, x) + t(y, x) + (x, ty) + t^2(y, y) = t^2(y, y) + 2t(x, y) + (x, x)
$$

Поскольку $x$ и $y$ мы зафиксировали и рассматриваем  функцию $\Phi(t)$ как функцию переменного $t$, то она квадратична. Также заметим, что она принимает неотрицательные значения, поскольку эта функция является скалярным квадратом.

Теперь рассмотрим значение этой функции как квадратный трехчлен и заметим, что его дискримант $\leq 0$, поскольку главный коэффициент $\geq 0$, как и значение квадратного трехчлена (график выше оси абцисс). Найдем значение дискриминанта 
$$
D = [2(x, y)]^2 - 4(y, y) \cdot (x, x) \leq 0
$$

$$
[(x, y)]^2 \leq (x, x) \cdot (y, y)
$$

$$
|(x, y)| \leq \sqrt{(x, x) \cdot (y, y)}
$$

$$
|(x, y)| \leq |x| \cdot |y|
$$
Полученное неравенство $|(x, y)| \leq |x| \cdot |y|$ - неравенство Коши-Буняковского.

Докажем выполнение 4-й аксиомы метрики евклидова пространства. Рассмотрим $\rho^2(x, y)$ и преобразуем выражение, воспользовавшись свойством скалярного произведения (1) и (2)
$$
\rho^2(x, y) = (x - y, x - y) = (x, x) - (y, x) - (x, y) + (y, y) = (x, x) - 2(x, y) + (y, y)
$$
Теперь сделаем оценку $-(x, y) \leq |(x, y)|$ и воспользуемся неравенством Коши-Буняковского
\begin{equation}
(x, x) - 2(x, y) + (y, y) \leq (x, x) + 2 \sqrt{(x, x)} \cdot \sqrt{(y, y)} + (y, y)
\end{equation}
Выражение $(x - y, x - y)$ представимо в виде $(x - z - (y - z), x - z - (y - z))$. Подставляем $x = x - z$ и $y = y - z$ в выражение $(1)$
\begin{equation*}
\begin{gathered}
\rho^2(x, y) \leq (x - z, x - z) + 2\sqrt{x - z, x - z} \cdot 2\sqrt{y - z, y - z} + (y - z, y - z) = \\
\rho^2(x, z) + 2\rho(x, z) \cdot \rho(y, z) + \rho^2(y, z) = [\rho(x, z) + \rho(y, z)]^2
\end{gathered}
\end{equation*}
Извлекаем корень из двух частей неравенства
$$
\rho(x, y) \leq \rho(x, z) + \rho(y, z)
$$
Выполнение 4-й аксиомы доказано.

\section{Последовательность в пространстве $\mathbb {R}^m$. Сходимость последовательности в метрическом пространстве. Основные свойства сходящихся последовательностей (ограниченность, единственность предела, сходимость подпоследовательности, арифметические операции). Критерий сходимости последовательности в $m$-мерном пространстве.}
\subsection{Последовательность в пространстве $\mathbb {R}^m$. Сходимость последовательности в метрическом пространстве.}
Последовательностью в $\mathbb {R}^m$ называется множество векторов в $\mathbb {R}^m$
$$
\textbf{x}^{(1)}, \textbf{x}^{(2)}, ..., \textbf{x}^{(n)}
$$
Обозначим последовательность векторов таким образом 
$$ \textbf{x}^{(n)} = (x^{(n)}_1, x^{(n)}_2, ..., x^{(n)}_m)$$
Где $\textbf{x}^{(n)}$ - векторы, $x^{(n)}_1, x^{(n)}_2, ..., x^{(n)}_m$ - координаты соответствующих векторов.

Последовательность$\{\textbf{x}^{(n)}\}^{\infty}_{n=1} \subset \mathbb {R}^m$ сходится, если 
$$
\exists \textbf {a} \in \mathbb {R}^m \; \forall \varepsilon > 0 \; \exists N_{\varepsilon} \in \mathbb {N} \; \forall n > N_{\varepsilon} \quad \rho(\textbf{x}^{(n)}, \textbf {a}) < {\varepsilon}, \quad \textbf {a} = \lim_{n\to\infty}{\textbf{x}^{(n)}}
$$
Можем переписать это в другом виде 
$$
\textbf {a} = \lim_{n\to\infty}{\textbf{x}^{(n)}} \Leftrightarrow \forall \varepsilon \geq 0 \; \exists N_{\varepsilon} \in \mathbb {N} \; \forall n > N_{\varepsilon} \quad \textbf{x}^{(n)} \in B(\textbf {a}, \varepsilon)
$$
То есть, начиная с некоторого номера, все элементы последовательности попадают в произвольный $\varepsilon$-шар с центром в точке $\textbf {a}$.
\subsection{Свойства сходящихся последовательностей}
\begin{enumerate} 
  \item 
  Если $\textbf {a} = \lim_{n\to\infty}{\textbf{x}^{(n)}}$, то за пределами любого $B(\textbf {a}, \varepsilon)$, может находится не более конечного числа элементов последовательности.
  \item
  Из первого свойства вытекает, что сходящаяся последовательность ограничена.
  \item
  Единственность предела.
  
  Предположим, что последовательность имеет два предела, тогда 
$$
\exists \textbf {a} \in \mathbb {R}^m \; \forall \varepsilon > 0 \; \exists N_{\varepsilon} \in \mathbb {N} \; \forall n > N_{\varepsilon} \quad \rho(\textbf{x}^{(n)}, \textbf {a}) < {\varepsilon}, \quad \textbf {a} = \lim_{n\to\infty}{\textbf{x}^{(n)}}
$$

$$
\exists \textbf {b} \in \mathbb {R}^m \; \forall \varepsilon > 0 \; \exists K_{\varepsilon} \in \mathbb {N} \; \forall k > K_{\varepsilon} \quad \rho(\textbf{x}^{(n)}, \textbf {b}) < {\varepsilon}, \quad \textbf {b} = \lim_{n\to\infty}{\textbf{x}^{(n)}}
$$
Пусть $B(\textbf {a}, \varepsilon)$ и $B(\textbf {b}, \varepsilon)$ не пересекающиеся $\varepsilon$-шары. Тогда имеем, что за пределами $B(\textbf {a}, \varepsilon)$ лежит конечное число элементов последовательности, и за пределами $B(\textbf {b}, \varepsilon)$ находится конечное число элементов последовательности. Приходим к противоречию.
  \item
  Если последовательность сходится к вектору $\textbf {a}$, то любая её подпоследовательность сходится $\textbf {a}$.
  $$
  \forall \{n_k\}^{\infty}_{k=1}, \;\textbf {x}^{(n)} \underset{k \to \infty}{\longrightarrow} \; \textbf {a} \Rightarrow \textbf {x}^{(n_k)} \underset{k \to \infty}{\longrightarrow} \; \textbf {a}
  $$
  $\{m_k\}^{\infty}_{k=1}$ - строго возрастающая последовательность 
  
  Доказательство:
  
  Поскольку последовательность $\textbf {x}^{(n)}$ сходится к $\textbf {a}$, то, по свойству сходящихся последовательностей (1), за пределами произвольного $\varepsilon$-шара с центром в точке $\textbf {a}$ находится конечное число элементов последовательности. Подпоследовательность получается путём "вычеркивания" некоторого ряда элементов из изначальной последовательности, следовательно, за пределами произвольного $\varepsilon$-шара с центром в точке $\textbf {a}$ находится конечное число элементов подпоследовательности, тогда в этом шаре лежат все элементы подпоследовательности, начиная с некоторого номера, что и означает наличие предела подпоследовательности в точке $\textbf {a}$.
  \item
  $$
  \textbf {x}^{(n)} \underset{n \to \infty}{\longrightarrow} \; \textbf {a} \; \Rightarrow |\textbf {x}^{(n)}| \underset{n \to \infty}{\longrightarrow} |\textbf {a}|
  $$
  Уточним, что слева идёт речь о сходимости в $\mathbb {R}^m$, а справа о сходимости в $\mathbb {R}$.
  
  Доказательство
  
  Воспользуемся определением модуля вектора
  \begin{equation}
  ||\textbf {x}^{(n)}| - |\textbf {a}|| = |\rho(\textbf {x}^{(n)}, 0) - \rho(\textbf {a}, 0)|
  \end{equation}
  Распишем $\rho(\textbf {x}^{(n)}, 0)$ по неравенству треугольника
  $$
  \rho(\textbf {x}^{(n)}, 0) \leq \rho(\textbf {x}^{(n)}, \textbf {a}) + \rho(\textbf {a}, 0) 
  $$
  Представим в другом виде
  $$
  \rho(\textbf {x}^{(n)}, 0) - \rho(\textbf {a}, 0) \leq \rho(\textbf {x}^{(n)}, \textbf {a}) \Rightarrow |\rho(\textbf {x}^{(n)}, 0) - \rho(\textbf {a}, 0)| \leq \rho(\textbf {x}^{(n)}, \textbf {a}) 
  $$
  Воспользуемся этой оценкой в выражении (2) и вспомним, что по условию ${\rho(\textbf {x}^{(n)}, \textbf {a}) \underset{n \to \infty}{\longrightarrow} 0}$
  $$
  ||\textbf {x}^{(n)}| - |\textbf {a}|| = |\rho(\textbf {x}^{(n)}, 0) - \rho(\textbf {a}, 0)| \leq \rho(\textbf {x}^{(n)}, \textbf {a}) \underset{n \to \infty}{\longrightarrow} 0
  $$
  Утверждение доказано.
  \item
  \begin{equation*}
  \begin{gathered}
  \textbf {x}^{(n)} \longrightarrow \textbf {a}, \; \textbf {y}^{(n)} \longrightarrow \textbf {b} \Rightarrow \forall \lambda, \mu \in \mathbb {R}\\
  \lambda \textbf {x}^{(n)} + \mu \textbf {y}^{(n)} \longrightarrow \lambda \textbf {a} + \mu \textbf {b}
  \end{gathered}
  \end{equation*}
  
  Доказательство
  
  Зафиксируем метрику $\rho_0(x, y) = \underset{\forall i=1, m}{\max}{|x_i - y_i|}$
  
  По условию $\textbf {x}^{(n)}$, $\textbf {y}^{(n)}$ - сходятся, то есть
  \begin{equation*}
  \begin{gathered}
  \forall \varepsilon > 0 \; \forall n > N_1(\varepsilon) \;\rho_0(\textbf {x}, \textbf {a}) = \underset{\forall i=1, m}{\max}{|{x}^{(n)}_i - {a}_i|} < \frac{\varepsilon}{2|\lambda|}\\
    \forall \varepsilon > 0 \; \forall n > N_2(\varepsilon) \;\rho_0(\textbf {y}, \textbf {b}) = \underset{\forall i=1, m}{\max}{|{y}^{(n)}_i - {b}_i|} < \frac{\varepsilon}{2|\mu|}
  \end{gathered}
  \end{equation*}
  
  Рассмотрим расстояние между векторами $ \lambda \textbf {x}^{(n)} + \mu \textbf {y}^{(n)}$ и $\lambda \textbf {a} + \mu \textbf {b}$
  $$
  \rho(\lambda \textbf {x}^{(n)} + \mu \textbf {y}^{(n)}, \lambda \textbf {a} + \mu \textbf {b}) = \underset{\forall i=1, m}{\max}{|(\lambda {x}^{(n)}_i + \mu {y}^{(n)}_i) - (\lambda {a}_i + \mu  {b}_i)|}
  $$
  
  Воспользуемся неравенством треугольника и вынесем константы
  $$
  \underset{\forall i=1, m}{\max}{|(\lambda {x}^{(n)}_i + \mu {y}^{(n)}_i) - (\lambda {a}_i + \mu {b}_i)|} \leq \underset{\forall i=1, m}{\max}{(|\lambda| |{x}^{(n)}_i - {a}_i| + |\mu| |{y}^{(n)}_i - {b}_i|)}
  $$
  Поскольку максимум суммы не превосходит суммы максимумов, имеем
  $$
  \underset{\forall i=1, m}{\max}{(|\lambda| |{x}^{(n)}_i - {a}_i| + |\mu| |{y}^{(n)}_i - {b}_i|)} \leq |\lambda| \underset{\forall i=1, m}{\max}{|{x}^{(n)}_i - {a}_i|} + |\mu| \underset{\forall i=1, m}{\max}{|{y}^{(n)}_i - {b}_i|}
  $$
  Заметим, что в правой части неравенства стоит $\rho(\textbf {x}^{(n)}, \textbf {a})$ и $\rho(\textbf {y}^{(n)}, \textbf {b} )$. Таким образом
  $$
  \forall n > \max\{N_1(\varepsilon), N_2(\varepsilon)\} \quad \rho(\lambda \textbf {x}^{(n)} + \mu \textbf {y}^{(n)}, \lambda \textbf {a} + \mu \textbf {b}) \leq |\lambda| \cdot \rho(\textbf {x}^{(n)}, \textbf {a} ) + |\mu| \cdot \rho(\textbf {y}^{(n)}, \textbf {b} ) < \varepsilon
  $$
  Утверждение доказано.
  \subsection{Критерий сходимости в $\mathbb {R}^m$}
  
  Последовательность векторов сходится к некоторому вектору $\textbf {a}$, тогда и только тогда, когда она сходится покоординатно к вектору $\textbf {a}$.
  $$
  \textbf {x}^{(n)}  \underset{n \to \infty}{\longrightarrow} \textbf {a} \Leftrightarrow \forall i = \overline{1,m} \quad x^{(n)}_i \longrightarrow a_i
  $$
  
  Доказательство
 
  Необходимость. Зафиксируем евклидову метрику. $\textbf {x}^{(n)}  \underset{n \to \infty}{\longrightarrow} \textbf {a}$ в $\mathbb {R}^m$, это значит что
  \begin{equation*}
  \begin{gathered}
  \forall \varepsilon > 0 \; \exists N_{\varepsilon} \in \mathbb {N} \; \forall n > N_{\varepsilon} \quad \rho_e(\textbf {x}^{(n)}, \textbf {a}) < \varepsilon\\
  \sqrt{(x_1 - a_1)^2 + (x_2 - a_2)^2 + ... + (x_m - a_m)^2} < \varepsilon\\
  \forall i = \overline{1,m} \quad |x_i - a_i| \leq \sqrt{\sum\limits_{i=1}^m (x_m - a_m)^2} < \varepsilon
  \end{gathered}
  \end{equation*}
  Таким образом
  $$
  \forall \varepsilon > 0 \; \exists N_{\varepsilon} \in \mathbb {N} \; \forall n > N_{\varepsilon} \; \forall i = \overline{1,m} \quad |x_i - a_i| < \varepsilon
  $$
  Что означает сходимость в $\mathbb {R}$
  
  Достаточность. Запишем условие сходимости i-й последовательности координат
  $$
  \forall i = \overline{1,m} \quad \forall \varepsilon > 0 \; \exists N_{\varepsilon, i} \in \mathbb {N} \; \forall n > N_{\varepsilon, i} \quad \Rightarrow |x^{(n)}_i - a_i| < \frac{\varepsilon}{\sqrt{m}}
  $$
  Выберем такой номер, начиная с которого выполняется это неравенство для каждой i-й последовательности координат
  $$
  N_{\varepsilon} = \underset{i=\overline{1, m}}{\max\{N_{\varepsilon, i}\}}
  $$
  Посчитаем расстояние между вектором $\textbf {x}^{(n)}$ и вектором $\textbf {a}$
  $$
  \forall n > N_{\varepsilon} \quad \rho(\textbf {x}^{(n)}, \textbf {a}) = \sqrt{\sum\limits_{i=1}^m (\textbf {x}^{(n)}_i - \textbf {a}_i)^2} < \sqrt{\sum\limits_{i=1}^m \frac{{\varepsilon}^2}{m}}
  $$
  Воспользуемся этой оценкой и вынесем константу, при этом под знаком суммы останется $m$ штук единиц
  $$
  \rho(\textbf {x}^{(n)}, \textbf {a}) < \sqrt{\sum\limits_{i=1}^m \frac{{\varepsilon}^2}{m}} = \sqrt{\frac{{\varepsilon}^2}{m}} \cdot \sqrt{m} = \varepsilon
  $$
\end{enumerate}