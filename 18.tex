\section{Дифференцируемость вектор-функции многихпеременных}
\subsection{Частные производные}

Пусть векторная функция нескольких переменных
\(f: \mathbb{R}^{n} \rightarrow \mathbb{R}^{m}\) определена в \(\delta\)
- окрестности \(\mathrm{U}(a, \delta)\) точки \(a \in \mathbb{R}^{n} .\) \(\forall b \in  \mathbb{R}^{n}: 0 < \rho(a,b) < \delta  \)
Обозначим через \(\Delta x_{i}\) такое приращение независимого
переменного \(x_{i}\) в точке \(a,\) при котором точка
\(a=\left(a_{1}, \ldots, a_{i-1}, a_{i}+\Delta x_{i}, a_{i+1}, \ldots, a_{n}\right)\)
принадлежит \(\mathrm{U}(a, \delta)\) Для этого достаточно, чтобы
выполнялось неравенство \(\left|\Delta x_{i}\right|<\delta .\) Тогда
определена разность значений функции \(f,\) соответствующая приращению
\(\Delta x_{i}\)

\[\Delta_{i} f\left(a, \Delta x_{i}\right)=f\left(a_{1}, \ldots, a_{i-1}, a_{i}+\Delta x_{i}, a_{i+1}, \ldots, a_{n}\right)-f\left(a_{1}, \ldots, a_{n}\right)\]

Эту разность называют \textbf{частным приращением функции нескольких
переменных} \(f\) в точке \(a\) по независимому переменному \(x_{i} .\)
Частное приращение обозначают также через \(\Delta_{i} f(a)\) или
\(\Delta_{x_{i}} f(a)\)

\subsubsection{Определение частной производной векторной функции нескольких переменных}

Если для функции нескольких переменных
\(f: \mathbb{R}^{n} \rightarrow \mathbb{R}^{m},\) определенной в
окрестности точки \(a,\) существует предел

\[\lim _{\Delta x_{i} \rightarrow 0} \frac{\Delta_{i} f(a)}{\Delta x_{i}}\]

отношения частного прирашения функции по переменному \(x_{i}\) к
приращению \(\Delta x_{i}\) этого же переменного при
\(\Delta x_{i} \rightarrow 0,\) то этот предел называют \textbf{частной
производной функции нескольких перменных} \(f\) в точке \(a\) по
переменному \(x_{i}\) и обозначают \(f_{x_{i}^{\prime}}^{\prime}\)

Тогда:

\[ f_{x_{i}^{\prime}}^{\prime}=\lim _{\Delta x_{i} \rightarrow 0} \frac{\Delta_{i} f(a)}{\Delta x_{i}}\]

\subsubsection{Теорема}

Для того чтобы векторная функция
\(f: \mathrm{U}(a, \delta) \subset \mathbb{R}^{n} \rightarrow \mathbb{R}^{m}\)
имела частную производную в точке \(a\) по переменному \(x_{i},\)
необходимо и достаточно, чтобы все ее координатные функции имели частную
производную в точке \(a\) по тому же переменному \(x_{i}\).

Пусть \(\Delta x_{i}-\) приращение независимого переменного \(x_{i}\) в
точке \(a .\) Тогда соответствуюшее прирашение функции \(f\) в точке
\(a\) можно записать в виде:

\[\begin{array}{c}
\Delta_{i} f(a)=\left(\begin{array}{c}
f_{1}\left(a_{1}, \ldots, a_{i-1}, a_{i}+\Delta x_{i}, a_{i+1}, \ldots, a_{n}\right) \\
\vdots \\
f_{m}\left(a_{1}, \ldots, a_{i-1}, a_{i}+\Delta x_{i}, a_{i+1}, \ldots, a_{n}\right)
\end{array}\right)-\left(\begin{array}{c}
f_{1}(a) \\
\vdots \\
f_{m}(a)
\end{array}\right)= \\
=\left(\begin{array}{c}
f_{1}\left(a_{1}, \ldots, a_{i-1}, a_{i}+\Delta x_{i}, a_{i+1}, \ldots, a_{n}\right)-f_{1}(a) \\
\vdots \\
f_{m}\left(a_{1}, \ldots, a_{i-1}, a_{i}+\Delta x_{i}, a_{i+1}, \ldots, a_{n}\right)-f_{m}(a)
\end{array}\right)=\left(\begin{array}{c}
\Delta_{i} f_{1}(a) \\
\vdots \\
\Delta_{i} f_{m}(a)
\end{array}\right)
\end{array}\]

По определению \textbf{частной производной}:

\[\frac{\partial f(a)}{\partial x_{i}}=\lim _{\Delta x_{i} \rightarrow 0} \frac{\Delta_{i} f(a)}{\Delta x_{i}}=\lim _{\Delta x_{i} \rightarrow 0}\left(\begin{array}{c}
\frac{\Delta_{i} f_{1}(a)}{\Delta x_{i}} \\
\vdots \\
\frac{\Delta_{i} f_{m}(a)}{\Delta x_{i}}
\end{array}\right)\]

Проеобразуем: \\
В силу сходимости в  $\mathbb{R}^{m}$ - это покоординатная сходимость: \\
\[\frac{\partial f(a)}{\partial x_{i}}=\left(\begin{array}{c}
\lim _{\Delta x_{i} \rightarrow 0} \frac{\Delta_{i} f_{1}(a)}{\Delta x_{i}} \\
\vdots \\
\lim _{\Delta x_{i} \rightarrow 0} \frac{\Delta_{i} f_{m}(a)}{\Delta x_{i}}
\end{array}\right)=\left(\begin{array}{c}
\frac{\partial f_{1}(a)}{\partial x_{i}} \\
\vdots \\
\frac{\partial f_{m}(a)}{\partial x_{i}}
\end{array}\right)\]

Доказано.
