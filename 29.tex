\section{Достаточные условия локального экстремума функции нескольких переменных.}
\subsection{Первоисточники:}
\parindent=0cm
Зорич том 1, страница 562

Кудрявцев том 2, страница 299

Ильин том 1, страница 528

Фихтенгольц том 1, страница 417

\subsection{Понятие критической точки:}

Понятие критической точки можно ввести для произвольной векторной функции $\mathbb{R}^n \to \mathbb{R}^m$, однако, пока будет достаточно ввести это понятие для скалярной функции. 

\begin{definition}
Точка $x_0 \in \mathbb{R}^n$ называется критической точкой функции $f: \mathbb{R}^n \to \mathbb{R}$, если функция $f$ имеет в этой точке дифференциал, представляющий из себя нулевое отображение.
\end{definition}
\subsection{Достаточное условие экстремума}
Пусть $f: \mathbb{R}^n \to \mathbb{R}$ дважды непрерывно дифференцируема в некоторой окрестности $x_0$, а $x_0$ - критическая точка. Эта функция может быть представлена формулой Тейлора:
$$f(x_1+h_1,x_2+h_2,...,x_n+h+n)= f(x_1,x_2,...,x_n)+$$
$$+ {1 \over 2!} \sum_{i,j=1}^{n}{{\partial^2 f \over \partial x_i \partial x_j}(x_0)h_i h_j} + o(|h|^2)$$
\begin{theorem}
Если квадратичная форма $\sum_{i,j=1}^{n}{{\partial^2 f \over \partial x_i \partial x_j}(x_0)h_i h_j}$ знакоопределена, то точка $x_0$ является точкой строгого экстремума функции $f$. Если же форма принимает различные знаки, то точка не явяется точкой экстремума.
\end{theorem}
\begin{proof}
Скажем для начала, для чего необходимо, чтобы $x_0$ была критической точкой. Дело в том, что доказательство основывается на том факте, что в точке функция должна иметь дифференциал, чтобы можно было разложить её по формуле Тейлора. Но, нам бы никогда не удалось показать, что в точке функция имеет экстремум, будь этот дифференциал ненулевым (необходимое условие экстремума).
Перепишем формулу тейлора другим образом:
$$f(x_0+h) - f(x) = {1 \over 2!|h|^2} (\sum_{i,j=1}^{n}{\partial^2 f \over \partial x_i \partial x_j}(x_0){h_i \over |h|} {h_j \over |h|}+o(1))$$

В силу того, что квадратичная форма $\sum_{i,j=1}^{n}{\partial^2 f \over \partial x_i \partial x_j}(x_0){h_i \over |h|} {h_j \over |h|}$ знакоопределена, можно выбрать такое отношение СТРОГОГО естественного порядка $\beta$, что $[F]\ \beta\ 0$ на всей области определения.
Обратим внимание на то, что квадратичная форма является функцией на пространстве всех приращений. То есть бесконечно-малым приращениям соответствуют $\delta$-окрестности точки ноль (в метрическом смысле).
Заметим, что квадратичная форма $F(h)$  является непрерывной функцией. Выберем теперь произвольную компактную окрестность точки 0 (замкнутый шар конечного вещественного радиуса) $B^n(0)$ в пространстве приращений, на котором определена квадратичная форма. На ней она достигает своего максимума и минимума $m_1, m_2$ в силу теоремы Вейештрасса. Среди двух этих точек найдётся одна такая $m$, для которой: $$\forall h \in B^n{(0)}: [F(h)]\ {\beta=}\  [F(m)]$$. 


Слагаемое $o(1)$ устроено так, что для любой величины $\varepsilon$ найдётся такая окрестность точки $0$, что это слагаемое $o(1)$ будет по модулю строго меньше $\varepsilon$.

То есть, можно найти теперь окрестность $O(0)$, в которой $|o(1)| < |F(m)|$. Отсюда важно сделать вывод, что прибавление $o(1)$ к $F(m)$ не меняет знка данной формы. (Проверяется непосредственно для каждого из строгих порядков)


Поэтому $[F(m)]\ \beta\ 0\Rightarrow [F(m) + o(1)]\ \beta\ 0$.

Но, в то же время: $[F(h)] \beta= [F(m)] \Rightarrow [F(h)+o(1)]\ \beta= [F(m)+o(1)]$(в силу инвариантности отношения естественного порядка по сдвигу).

Далее, воспользовавшись транзитивностью отношения порядка: $[F(h)+o(1)]\ \beta\  0$

$$f(x_0+h) - f(x) = {1 \over 2!|h|^2} (F(h)+o(1))$$

${1 \over 2!|h|^2}$ положительное, а значит никак не влияет на знак выражения в скобках.
Таким образом, нам удалось найти окрестность $O(0)$, в которой $[f(x_0 + h) - f(x)]\ \beta\ 0$ или $[f(x_0 + h)] \beta [f(x)]$, что подтверждает тот факт, что в этой точке функция имеет строгий экстремум отношения порядка $\beta$.

Если же форма принимает различные знаки, то есть является знакопеременной, верно то, (из курса алгебры), что в некоторой окрестности нуля она принимает как положительные, так и отрицательные значения. То есть, какую бы окрестность нуля мы не взяли, в ней найдутся всегда два таких $h$, что приращение по одному будет отрицательным, а под другому положительным, что противоречит возможности существования экстремума в точке.
\end{proof}
\begin{consequence}
Если форма $F > 0$, то точка $x_0$ является точкой минимума. Если форма $F < 0$, то точка $x_0$ является точкой максимума.
\end{consequence}л
