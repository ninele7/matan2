\documentclass{article}

%Russian-specific packages
%--------------------------------------
\usepackage[russian]{babel}
%--------------------------------------

%Hyphenation rules
%--------------------------------------
\usepackage{hyphenat}
%--------------------------------------

%Math
%--------------------------------------
\usepackage{mathtools}
\usepackage{mathrsfs}
\usepackage{amsthm}
\usepackage{amsmath}
\usepackage{amssymb}
\usepackage{amsfonts}
\usepackage{accents}
\usepackage{systeme}
\DeclareMathOperator{\rank}{rank}
\DeclareMathOperator{\im}{im}
\DeclareMathOperator{\sign}{sign}
\newcommand\aug{\fboxsep=-\fboxrule\!\!\!\fbox{\strut}\!\!\!}
%-------------------------------------- 

%Hyperlinks
%--------------------------------------
\usepackage[svgnames]{xcolor}
\usepackage[colorlinks=true, linkcolor=Blue, urlcolor=Blue]{hyperref}
%--------------------------------------

%Graphics
%--------------------------------------
\usepackage{graphicx}
\usepackage{wrapfig}
%--------------------------------------

%errors fix
%--------------------------------------
\usepackage{microtype}
\usepackage[a4paper,left=3cm,right=3cm,top=2cm,bottom=2cm]{geometry}
\usepackage{verbatim}
%--------------------------------------
\begin{document}

\section{Формула Тейлора для функций многих переменных.}

Формулы Тейлора и Маклорена. Если функция  $z=f(x,y)$ имеет в некоторой окрестности точки
$(x_0,y_0)$ непрерывные частные производные до (n+1)-го порядка включительно, то для любой точки $(x,y)$ из этой окрестности справедлива формула Тейлора n-го порядка:

$f(x,y)= f(x_\textup{0},y_\textup{0}) + \sum\limits_{k=1}^n \frac1{\textbf{k!}} \left((x-x_0)\frac{d}{dx}+(y-y_0)\frac{d}{dy}\right)^k \; f(x_0,y_0) + \boldsymbol{o}(\rho^n)$ , где

$
\rho = \sqrt{(x-x_0)^2+(y-y_0)^2}
$,

$
\left((x-x_0)\frac{d}{dx}+(y-y_0)\frac{d}{dy}\right)f(x_0,y_0) \equiv
$

$
\equiv (x-x_0)\frac{df(x_0,y_0)}{dx}+(y-y_0)\frac{df(x_0,y_0)}{dy}
$,

$
\left((x-x_0)\frac{d}{dx}+(y-y_0)\frac{d}{dy}\right)^2 f(x_0,y_0)\equiv (x-x_0)^2 \;\frac{d^2 f(x_0,y_0)}{dx^2} 
$

$
+2(x-x_0)(y-y_0)\;\frac{d^2 f(x_0,y_0)}{dxdy}+(y-y_0)^2 \;\frac{d^2f(x_0,y_0)}{dy^2}
$

и т.д. Формула Тейлора, записанная в окрестности точки (0,0) называется формулой Маклорена. Например, для функции двух переменных при n=2:

$
f(x,y)=f(0,0)+\frac{df(0,0)}{dx}x+\frac{df(0,0)}{dy}y+\frac{d^2 f(0,0)}{2dx^2}x^2+
$

$
\frac{d^2 f(0,0)}{dxdy}xy+\frac{d^2 f(0,0)}{2dy^2}y^2+\boldsymbol{o}(\rho^2)
$

Аппроксимация функции многочленом. Выражение

$
T_n(x,y)=f(x_0,y_0)+\sum\limits_{k=1}^n \frac1{\textbf{k!}} \left((x-x_0)\frac{d}{dx}+(y-y_0)\frac{d}{dy}\right)^k f(x_0,y_0)
$
называется многочленом Тейлора n-го порядка. Поскольку $
f(x,y)-T_n(x,y)=\boldsymbol{o}(\rho^n)
$
, то в окрестности точки функцию $
(x_0,y_0)
$
можно приближенно заменить, или, как говорят, аппроксимировать, ее многочленом Тейлора, т.е.
$
f(x,y)\approx T_n(x,y)
$
. Чем ближе точка
$
(x,y)
$
к точке
$
(x_0,y_0)
$
, тем выше точность такой аппроксимации; кроме того, точность возрастает с ростом n. Это означает, что, чем больше непрерывных производных имеет функция 
$
z=f(x,y)
$
 , тем точнее представляет ее многочлен Тейлора.
\end{document}