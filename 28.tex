\section{28. Необходимые условия локального экстр    емума функции нескольких переменных.}
\section*{Первоисточники:}
\parindent=0cm
Зорич том 1, страница 562

Кудрявцев том 2, страница 299

Ильин том 1, страница 528

Фихтенгольц том 1, страница 417

\section{Характер функции}
\parindent=1cm
Мы занимаемся изучением функций $\mathbb{R}^n \to \mathbb{R}^m$, однако почти все последующие понятия, связанные с экстремумами функций распространяются, по большей части, на скалярные функции, то есть функции $\mathbb{R}^n \to \mathbb{R}$.

Дело в том, что понятие экстремума основано на том факте, что множество $B$, на которое функция $f: A \to B$ отображает другое множество $A$, должно быть частично упорядоченным. А все элементы образа некоторой окрестности точки $x_0$, называемой точкой экстремума функции, должны быть сравнимы с образом данной точки.

Вообще, задать отношение линейного порядка можно на ЛЮБОМ множестве (теорема Цермело), однако все способы задания такого порядка носят специфический, контр-интуитивный характер уже даже в случае привычного $\mathbb{R}^2$. Можно задать, конечно, такое отношение частичного порядка на $\mathbb{R}^2$, что понятие экстремума будет применимо к векторной функции $f: \mathbb{R}^n \to \mathbb{R}^2$, однако всё это специфика, вроде "специальных метрик". 

На множествах $R^n, n>1$ нет естественного отношения частичного порядка, а потому понятие экстремума для векторных функций лишено интереса.
\section{Понятие экстремума для функции\linebreak $\mathbb{R}^n \to \mathbb{R}$}
\begin{ftdef}
Точка $x_0$ метрического пространства $(\mathbb{R}^n, \rho)$\linebreak называется точкой локального экстремумума функции $f: \mathbb{R}^n \to \mathbb{R}$, если существует проколотая окрестность этой точки $X \in \mathbb{R}^n$, все элементы $x \in f(X)$ образа которой состоят в некотором естественном отношении порядка\linebreak $(f(x), f(x_0))$ с образом точки $x_0$.

Под естественным отношением порядка на $R$ понимается одно из следующих четырёх: $\leq, \geq, <, >$.
\end{ftdef}

\begin{ftdef}
Для отношений порядка $\leq, \geq, <, >$ точки\linebreak экстремума называются соответственно:

$\leq$ -- максимума
$\geq$ -- минимума

$<$ -- строгого максимума
$>$ -- строгого минимума

\end{ftdef}
\section{Необходимое условие экстремума.}
\parindent=0cm
Пусть $\alpha$ -- естественное отношение порядка на $\mathbb{R}^n$ ($\leq, \geq, <, >$).

И некоторая функция $f: \mathbb{R}^n \to \mathbb{R}$ имеет локальный экстремум порядка $\alpha$ в точке $x_0$.

Тогда верна следующая
\begin{theorem}
Если заданная функция имеет производную по некоторому вектору $v \in \mathbb{R}^n$ в точке $x_0$ своего экстремума, то производная эта равна нулю.
\end{theorem}
\begin{proof}


Докажем от противного. 
По определению\linebreak производной по вектору:
$$D_{v}f(x_0) = \lim_{t \to 0}{f(x_0 + vt)-f(x_0) \over t}$$
Предположим теперь, что такой предел не равен нулю, а равен $A \not = 0$.

$$ \lim_{t \to +0}{f(x_0 + vt)-f(x_0) \over t} = \lim_{t \to -0}{f(x_0 + vt)-f(x_0) \over t}=A$$
Заметим теперь, что $f(x_0 +vt)\ \alpha\  f(x_0)$, начиная с некоторого момента стремления $t$ к нулю, в силу определения локального экстремума.
Из этого, в частности, следует, что: ${f(x_0 + vt)-f(x_0) \over t}\ \alpha\ 0$, если $t>0$, и  $0\ \alpha\ {f(x_0 + vt)-f(x_0) \over t}$ при $t<0$ (используются свойства присущие всем естественным отношениям порядка на $\mathbb{R}$).

В силу равенства пределов при $t\to 0$ данного выражения, как при $t<0$, так и при $t>0$, а также в силу отделимости от нуля, это выражение, начиная с некоторого момента ($\delta$-окрестность в $\mathbb{R}$), имеет одинаковый знак, совпадающий со знаком $A$, для любого $t$ из этой окрестности, при этом сравниваясь с нулём по порядку $\alpha$ различным образом с обеих сторон. Но, совершенно ясно, что одинаковость знака двух выражений $a,b$ влечёт либо $a\ \alpha\ 0,\ b\ \alpha\ 0$, либо $0\ \alpha\ a,\ 0\ \alpha\ b$,но никак не $a\ \alpha\ 0,\ 0\ \alpha\ b$, за исключением случая, когда оба выражения равны нулю, но у нас $A \not =0$. Получаем противоречие.
\end{proof}
\begin{consequence}
Если функция $f: \mathbb{R}^n \to \mathbb{R}$, имеет в точке $x_0$ локального экстремума частную производную по $i-ой$ координате, эта производная равна 0.$$\exists {\partial f \over \partial x_i}(x_0) \Rightarrow {\partial f \over \partial x_i}(x_0) = 0$$
\end{consequence}
\begin{consequence}
Если функция $f: \mathbb{R}^n \to \mathbb{R}$ дифференцируема в точке  локального экстремума $x_0$, то выполнены следующие свойства: 

1) Дифференциал в этой точке представляет из себя линейную функцию $df(x_0)$, такую, что $\forall x \in \mathbb{R}^n: df(x_0)(x) = 0$ 

2) Вектор градиента $\nabla f(x_0) \in \mathbb{R}^n$ представляет из себя нулевой вектор. 

3) Касательная плоскость к графику функции $z = f(x)$ в точке $(x_0*,f(x_0)) \in \mathbb{R}^{n+1}$ имеет следующее уравнение: $z = f(x_0)$, где $x_0*$ -- координаты точки $x_0$ в стандартном базисе. То есть, в случае когда $f: \mathbb{R}^n \to \mathbb{R}$, такая касательная плоскость лежит горизонтально.
\end{consequence}

