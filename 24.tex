\section{Производные второго и более высоких порядков.}
Пусть дана функция \(f(x_1,x_2,\cdots,x_m)\), \(f\) определена на \(X \subseteq \mathbb{R}^m\),
и данная функция имеет частную производную \(\frac{\partial f}{\partial x_i}(x_1,x_2,\cdots,x_m)\)
в некоторой окрестности некоторой точки\\ \(M(x_1,x_2,\cdots,x_m)\).\\\\
Обозначим частную производную как некоторую функцию:
\begin{equation*}
    \frac{\partial f}{\partial x_i}(x_1,x_2,\cdots,x_m) = g(x_1,x_2,\cdots,x_m)
\end{equation*}
Теперь дадим частное прирощение в точке \(M\) по некоторой координате \(x_k\):
\begin{equation*}
    M(x_1,x_2,\cdots,x_k, x_{k+1},\cdots,x_m) \to N(x_1,x_2,\cdots,x_k + \Delta x_k, x_{k+1},\cdots,x_m)
\end{equation*}
Сосчитаем предел:
\begin{equation*}
    \lim_{\substack{\Delta x_k \to 0 \\ N \to M}} \frac{g(N) - g(M)}{\Delta x_k}
\end{equation*}
Если такой предел существует, то он является частной производной функции \(g\) по переменной \(x_k\):
\begin{equation*}
    \lim_{\substack{\Delta x_k \to 0 \\ N \to M}} \frac{g(N) - g(M)}{\Delta x_k}
    = \frac{\partial g}{\partial x_k} (M)
    = \frac{\partial}{\partial x_k} \left( \frac{\partial f}{\partial x_i} \right) (M)
    = \frac{\partial^2 f}{\partial x_k \partial x_i} (M)
\end{equation*}
Выражение \(\frac{\partial^2 f}{\partial x_k \partial x_i} (M)\) называется производной второго порядка
функции \(f\) по переменным \(x_i, x_k\).\\\\
В частности, если \(k = i\), то:
\begin{equation*}
    \frac{\partial^2 f}{\partial x_i \partial x_i} (M) = \frac{\partial^2 f}{\partial x_i^2} (M)
\end{equation*}
И \(\frac{\partial^2 f}{\partial x_i^2} (M)\) называется второй частной производной функции \(f\)
по переменной \(x_i\).\\\\
Если \(k \neq i\), то \(\frac{\partial^2 f}{\partial x_k \partial x_i} (M)\) называется смешанной производной
второго порядка функции \(f\) по переменным \(x_i, x_k\).\\
\begin{equation*}
    \frac{\partial^2 f}{\partial x_k \partial x_i} = \left( f'_{x_i} \right)'_{x_k} = f''_{x_i x_k}
\end{equation*}