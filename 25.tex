\section{Достаточные условия равенства смешанных производных}
Прежде чем доказывать 3 основные теоремы о достаточных условиях равенства смешанных производных введём вспомогательное  выражение: 
$$\Phi = f(x_0 + \Delta x, y_0 + \Delta y) - f(x_0 + \Delta x, y_0) - f(x_0, y_0 + \Delta y) + f(x_0, y_0)$$

Введём также функцию:
$$
\phi (x) = f(x, y_0 + \Delta y) -  f(x, y_0)
$$
Тогда:
$$\Phi = \phi (x_0 + \Delta x) - \phi (x_0)$$
применим теорему Лагранжа:
$$
\Phi = \phi^{'}(x_0 + \Theta_1\Delta x)\Delta x = [f_x^{'}(x_0 + \Theta_1\Delta x, y_0 + \Delta y) - f_x^{'}(x_0 + \Theta_1\Delta x, y_0)]\Delta x
$$  
по теореме Лагранжа:
$$
\Phi = f_{x y}^{''}(x_0 + \Theta_1\Delta x, y_0 + \Theta_2\Delta y)\Delta y \Delta x
$$
Проделыйваем то же самое с $y$: 
$$
\psi (y) = f(x_0 + \Delta x, y) -  f(x_0, y)
$$ 
Тогда:
$$
\Phi = \psi (y_0 + \Delta y) - \psi (y_0)
$$
Далее, дважды применяя теорему Лагранжа:
$$
\Phi = [f_y^{'}(x_0 + \Delta x, y_0 + \lambda_1\Delta y) - f_y^{'}(x_0, y_0+\lambda_1\Delta y)]\Delta y = f_{y x}^{''}(x_0 + \lambda_1\Delta x, y_0 + \lambda_2\Delta y)\Delta y \Delta x
$$
Перейдем к теоремам:
 
Теорема 1: Пусть в $O(x_0, y_0)$ определены и непрерывны $f, f_{x}^{'}, f_{y}^{'}$, определены $f_{x y}^{''}, f_{y x}^{''}$. Если $f_{x y}^{''}, f_{y x}^{''}$ непрерывны в $(x_0, y_0)$, то $f_{x y}^{''}(x_0, y_0) = f_{y x}^{''}(x_0, y_0)$
 
Возьмём приращения $\Delta x, \Delta y$, оставляющие нас в рамках $O(x_0, y_0)$ и поделим на них выражение $\Phi$:
$$
\frac{\Phi}{\Delta x \Delta y} = f_{x y}^{''}(x_0 + \Theta_1\Delta x, y_0 + \Theta_2\Delta y) \underset{\Delta x, \Delta y \to 0}{\longrightarrow} f_{x y}^{''}(x_0, y_0)
$$
$$
\frac{\Phi}{\Delta x \Delta y} = f_{y x}^{''}(x_0 + \lambda_1\Delta x, y_0 + \lambda_2\Delta y) \underset{\Delta x, \Delta y \to 0}{\longrightarrow} f_{y x}^{''}(x_0, y_0)
$$

Теорема 2:  Пусть в $O(x_0, y_0)$ определены и непрерывны $f, f_{x}^{'}, f_{y}^{'}$.

Пусть в $\mathring O(x_0, y_0)$ определена хотя бы одна из $f_{y x}^{''}  f_{x y}^{''} $

Пусть наконец $\exists \lim_{x \to x_0, y \to y_0}{f_{x y}^{''}}$. 

Тогда 
$\exists f_{x y}^{''}(x_0, y_0), f_{y x}^{''}(x_0, y_0) \And f_{x y}^{''}(x_0, y_0) = f_{y x}^{''}(x_0, y_0)$.

$$
\lim_{\Delta x \to 0}{\frac{\Phi}{\Delta x \Delta y}} = 
\lim_{\Delta x \to 0}{\frac{[f_x^{'}(x_0 + \Theta_1\Delta x, y_0 + \Delta y) - f_x^{'}(x_0 + \Theta_2\Delta x, y_0)]\Delta x}{\Delta x \Delta y}} = \frac{1}{\Delta y}[f_x^{'}(x_0, y_0 + \Delta y) - f_x^{'}(x_0, y_0)]
$$ 
(В силу непрерывности первой производной в точке $(x_0, y_0 + \Delta y)$ и $(x_0, y_0)$)
проворачиваем аналогично с $\Delta y$: 
$$
\lim_{\Delta x \to 0}{\frac{\Phi}{\Delta x \Delta y}} = \lim_{\Delta y \to 0}{\frac{[f_y^{'}(x_0 + \Delta x, y_0 + \lambda_1\Delta y) - f_y^{'}(x_0, y_0 + \lambda_2\Delta y)]\Delta y}{\Delta x \Delta y}} = \frac{1}{\Delta y}[f_y^{'}(x_0 + \Delta x, y_0) - f_y^{'}(x_0, y_0)]
$$

теперь устремим $\Delta x$ и $\Delta y$ одновременно к 0 и тогда по теореме о перестановке двух повторных пределов:
$$
\lim_{\Delta x, \Delta y \to 0}{\frac{\Phi}{\Delta x \Delta y}} = \lim_{\Delta x, \Delta y \to 0}{f_{x y}^{''}(x_0 + \Theta_1\Delta x, y_0 + \Theta_2\Delta y}) = A = f_{x y}^{''}(x_0, y_0)
$$
$$
\lim_{\Delta y \to 0}{\lim_{\Delta x \to 0}{\frac{\Phi}{\Delta x \Delta y}}} = A = f_{x y}^{''}(x_0, y_0) = \lim_{\Delta x \to 0}{\lim_{\Delta y \to 0}{\frac{\Phi}{\Delta x \Delta y}}} = f_{y x}^{''}(x_0, y_0)
$$

Дадим вспомогательное определение: Функция является дважды дифференцируемой в точке $x^0$, если все её частные производные первого порядка являются функциями дифференцируемыми в точке $x_0$

Теорема 3: Пусть функция $f$ определена в $O(x_0, y_0)$ и дважды диффиренцируема в $(x_0, y_0)$. Тогда $f_{x y}^{''}(x_0, y_0) = f_{y x}^{''}(x_0, y_0)$

$\Phi = [f_x^{'}(x_0 + \Theta_1\Delta x, y_0 + \Delta y) + f_{x}^{'}(x_0, y_0) - f_{x}^{'}(x_0, y_0) - f_x^{'}(x_0 + \Theta_1\Delta x, y_0)]\Delta x = [f_x^{'}(x_0 + \Theta_1\Delta x, y_0 + \Delta y) - f_{x}^{'}(x_0, y_0)]\Delta x - [f_x^{'}(x_0 + \Theta_1\Delta x, y_0) - f_{x}^{'}(x_0, y_0)]\Delta x =  [f_{x x}^{''}(x_0, y_0)\Theta_1\Delta x + f_{y y}^{''}(x_0, y_0)\Delta y + \epsilon_1\Theta_1\Delta x + \epsilon_2 \Delta y] \Delta x - [f_{x x}^{''}(x_0, y_0)\Theta_1\Delta x + \epsilon_3\Theta_1\Delta x] \Delta x = 
f_{x y}^{''}(x_0, y_0)\Delta x\Delta y + \epsilon_1\Theta_1(\Delta x)^2 + \epsilon_2 \Delta x\Delta y - \epsilon_3\Theta_1(\Delta x)^2$

$\Phi = [f_y^{'}(x_0 + \Delta x, y_0 + \lambda_1\Delta y) + f_{y}^{'}(x_0, y_0) - f_{y}^{'}(x_0, y_0) - f_y^{'}(x_0, y_0+ \lambda_2\Delta y)]\Delta y = [f_{y x}^{''}(x_0, y_0)\Delta x + f_{y y}^{''}(x_0, y_0)\lambda_1\Delta y + \delta_1\Delta x + \delta_2\lambda_1\Delta y]\Delta y - [f_{y y}^{''}(x_0, y_0)\lambda_1\Delta y+ \delta_3\lambda_1\Delta y]\Delta y = f_{y x}^{''}(x_0, y_0)\Delta x\Delta y + \delta_1\Delta x\Delta y + \delta_2\lambda_1(\Delta y)^2 - \delta_3\lambda_1(\Delta y)^2$

Положим $\Delta y = \Delta x$ и тогда:

$$
\frac{\Phi}{(\Delta x)^2} = f_{x y}^{''}(x_0, y_0) + \epsilon_1\Theta_1 + \epsilon_2  - \epsilon_3\Theta_1\underset{\Delta x \to 0}{\longrightarrow} f_{x y}^{''}(x_0, y_0)
$$ 

$$
\frac{\Phi}{(\Delta x)^2} = f_{y x}^{''}(x_0, y_0) + \delta_1 + \delta_2\lambda_1 - \delta_3\lambda_1 \underset{\Delta x \to 0}{\longrightarrow} f_{y x}^{''}(x_0, y_0)
$$
