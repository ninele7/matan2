\section{Дифференциал второго и более порядков}
Дифференциалом $n$-ого порядка, $n \geq 2$, функции $z=f(x_1, x_2, \dots, x_k)$ называется:
$$d(d^{n-1}z)$$
\subsection{Независимые переменные}
При $n = 2$, и при условии что переменные $x_i$ независимы:
$$d^2z = \left(
{\partial^2z \over \partial x_1^2}d x_1 +
{\partial^2z \over \partial x_1\partial x_2}d x_2 +
\dots +
{\partial^2z \over \partial x_1\partial x_k}d x_k  
\right)d x_1 + $$
$$ + \left(
{\partial^2z \over \partial x_2\partial x_1}d x_1 +
{\partial^2z \over \partial x_2^2}d x_2 +
\dots +
{\partial^2z \over \partial x_2\partial x_k}d x_k  
\right)d x_2 + $$
$$ + \left({\partial^2z \over \partial x_k\partial x_1}d x_1 +
{\partial^2z \over \partial x_k\partial x_2}d x_2 +
\dots +
{\partial^2z \over \partial x_k^2}d x_k  
\right)d x_k = $$
$$= {\partial^2z \over \partial x_1^2}d x_1^2 +
{\partial^2z \over \partial x_2^2}d x_2^2 +
\dots  +
{\partial^2z \over \partial x_k^2}d x_k^2 +  $$
$$+ 2 {\partial^2z \over \partial x_1x_2}dx_1dx_2 + 
2{\partial^2z \over \partial x_1x_3}dx_1dx_3 +
\dots
2{\partial^2z \over \partial x_1x_k}dx_1dx_k +
\dots  +
2{\partial^2z \over \partial x_{k-1}x_k}dx_{k-1}dx_k$$

Символически, дифференциал первого порядка можно записать, вынесев функцию за скобку:
$$dz = \left(
{\partial \over \partial x_1}dx_1 +
{\partial \over \partial x_2}dx_2 +
\dots +
{\partial \over \partial x_k}dx_k
\right)z$$
Можно заметить, что если вынести функцию за скобку в выражении дифференциала второго порядка, то в скобках останется 
$$\left(
{\partial \over \partial x_1}dx_1 +
{\partial \over \partial x_2}dx_2 +
\dots +
{\partial \over \partial x_k}dx_k
\right)^2$$
т.е. дифференциал второго порядка можно записать как
$$d^2z = \left(
{\partial \over \partial x_1}dx_1 +
{\partial \over \partial x_2}dx_2 +
\dots +
{\partial \over \partial x_k}dx_k
\right)^2z$$
"Правило" заключается в том, что сначала по правилам алгебры раскрывается степень, а затем все полученные члены домнажаются на $z$.
Докажем что это правило работает для $d^nz$, если оно работает для $d^{n-1}z$.
\begin{proof}
	Мы уже знаем, что правило работает при $n =1$, $2$, поэтому осталось доказать шаг индукции.\newline
	$$d^nz = \Sigma
	C_{
		\alpha_1,
		\alpha_2
		\dots,
		\alpha_k
		}
	\cdot
	{\partial^nz \over 
	\partial x_1^{\alpha_1}
	\partial x_2^{\alpha_2}
	\dots
	\partial x_k^{\alpha_k}
	}
	\cdot
	dx_1^{\alpha_1}
	dx_2^{\alpha_2}
	\dots
	dx_k^{\alpha_k}
	$$
Где $
		\alpha_1,
		\alpha_2
		\dots,
		\alpha_k$ - всевозможные группы неотрицательных чисел удовлетворяющих условию $ \alpha_1 + \alpha_2 + \dots + \alpha_k = n$, а
$$
	C_{
		\alpha_1,
		\alpha_2
		\dots,
		\alpha_k
	} = 
	{n! \over
	\alpha_1!
	\alpha_2!
	\dots
	\alpha_k!}
	$$
Суть "полиномиальные" коэффициенты.
	\newline
	\newline
Предполагая что существует дифференциал $n + 1$, продифференцируем предыдущее выражение:
	\begin{align*}d^{n+1}z = \Sigma
	C_{
		\alpha_1,
		\alpha_2
		\dots,
		\alpha_k
		}
		\cdot\Big(
	{\partial^{n+1}z \over 
	\partial x_1^{\alpha_1 + 1}
	\partial x_2^{\alpha_2}
	\dots
	\partial x_k^{\alpha_k}
	}
	\cdot
	dx_1^{\alpha_1 + 1}
	dx_2^{\alpha_2}
	\dots
		dx_k^{\alpha_k} &+\\+
		{\partial^{n+1}z \over 
	\partial x_1^{\alpha_1}
	\partial x_2^{\alpha_2 + 1}
	\dots
	\partial x_k^{\alpha_k}
	}
	\cdot
	dx_1^{\alpha_1}
	dx_2^{\alpha_2 + 1}
	\dots
	dx_k^{\alpha_k}
		&+\\+
		\dots 
		&+\\+
		{\partial^{n+1}z \over 
	\partial x_1^{\alpha_1}
	\partial x_2^{\alpha_2}
	\dots
	\partial x_k^{\alpha_k + 1}
	}
	\cdot
	dx_1^{\alpha_1}
	dx_2^{\alpha_2}
	\dots
	dx_k^{\alpha_k + 1}\Big)
	\end{align*}
	То же самое выражение мы могли бы получить, формально перемножив символические выражения
	$$\Sigma
	C_{
		\alpha_1,
		\alpha_2
		\dots,
		\alpha_k
		}
	\cdot
	{\partial^n \over 
	\partial x_1^{\alpha_1}
	\partial x_2^{\alpha_2}
	\dots
	\partial x_k^{\alpha_k}
	}
	\cdot
	dx_1^{\alpha_1}
	dx_2^{\alpha_2}
	\dots
	dx_k^{\alpha_k} \cdot
	\left(
	{\partial \over \partial x_1}dx_1 + 
	{\partial \over \partial x_2}dx_2 + 
	\dots
	{\partial \over \partial x_k}dx_k
	\right)
	$$
	И приписав к ним $z$.
	Но это "произведение" суть
	\begin{align*}
	\left(
	{\partial \over \partial x_1}dx_1 + 
	{\partial \over \partial x_2}dx_2 + 
	\dots
	{\partial \over \partial x_k}dx_k
	\right)^k
	\cdot
		\left(
	{\partial \over \partial x_1}dx_1 + 
	{\partial \over \partial x_2}dx_2 + 
	\dots
	{\partial \over \partial x_k}dx_k
	\right) = \\
		\left(
	{\partial \over \partial x_1}dx_1 + 
	{\partial \over \partial x_2}dx_2 + 
	\dots
	{\partial \over \partial x_k}dx_k
		\right)^{k+1}
	\end{align*}
	Что и требовалось доказать.
\end{proof}
\subsection{Зависимые переменные}
Пусть мы имеем сложную функцию $z = (x_1, x_2, \dots, x_k)$, где 
$$x_i = \phi(t_1, t_2, \dots, t_m), \quad i = 1, 2, \dots, n$$
$dx_1, dx_2, \dots, dx_k$ - дифференциалы не независимых переменных, а значит сами являются функциями и могут не быть постоянными.
$$d^2z = 
d\left({\partial z \over \partial x_1}\right)dx_1 + 
d\left({\partial z \over \partial x_2}\right)dx_2 + 
\dots +
d\left({\partial z \over \partial x_k}\right)dx_k + 
$$
$$ + 
{\partial z \over \partial x_1}d(dx_1) + 
{\partial z \over \partial x_2}d(dx_2) + 
\dots +
{\partial z \over \partial x_k}d(dx_k)
$$
\subsection{Линейно зависимые переменные}
Рассмотрим случай, когда $x_1, x_2, \dots, x_k$ - линейные
функции от $t_1, t_2, \dots, t_m$, т.е.
$$x_i =
\alpha_{i1}t_1 + 
\alpha_{i2}t_2 + 
\dots +
\alpha_{im}t_m + 
\beta_i
\quad i = 1, 2, ... k$$
В этом случае
$$dx_i = 
\alpha_{i1}dt_1 + 
\alpha_{i2}dt_2 + 
\dots +
\alpha_{im}dt_m
= 
\alpha_{i1}\Delta t_1 + 
\alpha_{i2}\Delta t_2 + 
\dots +
\alpha_{im}\Delta t_m
= \Delta x_i
$$
все первые дифференциалы функций $x_1, x_2, \dots x_k$ постоянны (т.к. равны приращению) и не за
висят от $t_1, t_2, \dots, t_m$, а значит выкладки для функций с независи
мыми переменными работают и для функций с линейно зависимыми.
