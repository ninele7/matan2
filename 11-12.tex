\section{Определения непрерывности функции многих переменных по Коши и по Гейне.
Непрерывность на множестве.
Непрерывность в изолированной точке.}
\subsection{Определения непрерывности по Коши и по Гейне}
Пусть задана некоторая область $D \subseteq \mathbb{R}^m$ и для $ \forall x \in D \rightarrow \ ! y \in \mathbb{R}^{k}$, тогда функция определена на $D$, и $y=f(x)=f(x_{1}, x_{2}, ... , x_{n})$. Говорят, что функция $f$ - непрерывна в $a \in D$, если
$$
\exists \lim_{\substack{x_{1}\to a_{1}\\ ... \\{x_{n}\to a_{n}}}}{f(x_{1},..., x_{n})} = f(a_{1},..., a_{n})
$$
\textbf{по Гейне:}
$$
\forall x^{(n)} \in D, \ x^{(n)} \overset{\mathbb{R}^{m}}{\underset{n \rightarrow \infty}{\longrightarrow}} a \ \Rightarrow \
y^{(n)} = f(x^{(n)}) \rightarrow f(a) \in \mathbb{R}^{k}
$$
это означает, что какую бы последовательность векторов, сходящуюся к $a$, ни взять, соответствующая последовательность значений функции
$
f(x^{(1)}), f(x^{(2)}), ..., f(x^{(n)})
$
сходится к $f(a)$. \\
Стоит также уточнить, что если f - скалярная функция, то $f(a) \in \mathbb{R}$, а если f - вектор-функция, то есть
$$
f: D \in \mathbb{R}^{n} \rightarrow \mathbb{R}^{k}
$$
$$
x = (x_{1}, x_{2}, ... , x_{n}) \rightarrow y = (y_{1}, y_{2}, ... , y_{k})
$$
где
\begin{equation*}
y =
\begin{cases}
y_{1} =y_{1}(x_{1}, x_{2}, ... , x_{n}) \in \mathbb{R}\\
... \\
y_{k} =y_{k}(x_{1}, x_{2}, ... , x_{n}) \in \mathbb{R}
\end{cases}
\end{equation*}
- система k скалярных функций, то $f(a) \in \mathbb{R}^{k}$
\\\\
\textbf{по Коши:}
$$
\forall \varepsilon>0 \quad \exists \delta(\varepsilon)>0 :\ \forall x \in D \cap B(a, \delta)
\Rightarrow
$$
для скалярной функции: $|f(x) - f(a)| < \varepsilon$,
для вектор-функции: $f(x) \in B(a, \varepsilon)$.\\
Также можно записать и на языке метрик:
$$
\forall \varepsilon>0 \quad \exists \delta(\varepsilon)>0 \ \forall x \in D :\ \rho^{m}(x,a) < \delta
\Rightarrow \rho^{k}(f(x), f(a)) < \varepsilon
$$
\subsection{Непрерывность на множестве}
$f$ - непрерывна на некотором множестве $X$, если $f$ - непрерывна в каждой точке множества $X$

\subsection{Непрерывность в изолированной точке}
Всякая функция непрерывна в каждой изолированной точке множества своего определения.

\section{Основные свойства непрерывных функций. Устойчивость знака непрерывных функций}
\subsection{Основные свойства непрерывных функций.}
1. $f + g$ - непрерывна (в $\mathbb{R}^{k}$)\\
2. $f - g$ - непрерывна (в $\mathbb{R}^{k}$)\\
3. $f \cdot g$ - непрерывна (только в $\mathbb{R}$)\\
4. $f \over g$ - непрерывна (только в $\mathbb{R}$, если $g \neq 0 $) \\
5. сохранение знака непрерывной функции (только в $\mathbb{R}$)

\subsection{Устойчивость знака непрерывных функций.}
\textbf{Формулировка:}
Если $f(a) > 0 \ (f(a) < 0 )$ и $f$ - непрерывна в точке $a$, то найдутся такие $O(a)$ и $r>0$, что\\
$$
f(x) \geq r > 0 \ (f(x) \leq -r < 0)
$$
при $x \in O(a) \cap X$\\
\textbf{Доказательство:}\\
Если $a \in X$, то $\lim_{x\to a}{f(x)} = f(a)$, т.е
$$
\forall \varepsilon>0 \quad \exists \delta(\varepsilon)>0 \ \forall x \in O_{\delta}(a) \cap X
\Rightarrow |f(x)- f(a)| < \varepsilon
$$
Пусть $f(a)>0$. Положим $\varepsilon$ = ${f(a)\over 2} > 0$, по нему найдем $\delta(\varepsilon)$ и для любого $x \in O_{\delta}(a) \cap X$ получим
\begin{equation*}
\begin{gathered}
|f(x) - f(a)| < {f(a)\over 2}\\
f(a) - {f(a)\over 2} < f(x) < {3\over 2} f(a)\\
\end{gathered}
\end{equation*}
откуда
$$
f(x)> r = {f(a)\over 2} > 0
$$