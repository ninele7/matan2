\section{Критерий Коши.}

$\{x^{(n)}\}_{n=1}^{\infty}$ - \textcolor{red} {фундаментальная} в $\mathbb {R}^m$, если $\forall \varepsilon > 0$ $\exists n_\varepsilon \in \mathbb {N}$ : $\forall n, m > n_\varepsilon$

$\Rightarrow$ $\rho (x^{(n)}, x^{(m)})  < \varepsilon$

\subsection{Полнота пространства $\mathbb {R}^m$ }
Метрическое пространство ${(X, \rho)}$ называется \textcolor{red}{полным}, если любая фундаментальная последовательность сходится. 

В любом метрическом пространстве любая сходящаяся последвательность является фундаментальной. Действительно, если последовательность $\{x^{(n)}\}$ сходится, то 

$\exists a \in X:$ $\forall \varepsilon > 0$ $ \exists n_\varepsilon \in \mathbb {N}$ $\forall n > n_\varepsilon $ $\Rightarrow$ $\rho(x^{(n)}, a) < \frac{\varepsilon}{2}$

$\forall n, m > n_\varepsilon$: $\rho(x^{(n)}, x^{(m)}) < \rho(x^{(n)}, a) + \rho(a, x^{(m)}) < \frac{\varepsilon}{2} + \frac{\varepsilon}{2} = \varepsilon$

\subsection{Критерий Коши} 
$\{x^{(n)}\}_{n=1}^{\infty}$ - сходится в $\mathbb {R}^m$ $\Leftrightarrow$ $\{x^{(n)}\}_{n=1}^{\infty}$ - фундаментальная в $\mathbb {R}^m$.

\textcolor{red}{Доказательство}:

1. Из сходимости: $\forall \varepsilon > 0$ $ \exists n_\varepsilon \in \mathbb {N}$ $\forall n > n_\varepsilon $ $\Rightarrow$ $\rho(x^{(n)}, a) < \frac{\varepsilon}{2}$

$\forall \varepsilon > 0$ $\exists n_\varepsilon \in \mathbb {N}$ $\forall m > n_\varepsilon $ $\Rightarrow$ $\rho(x^{(m)}, a) < \frac{\varepsilon}{2}$


$\rho(x^{(n)}, x^{(m)})$ < $\rho(x^{(n)}, a) + \rho(x^{(m)}, a) = \frac{\varepsilon}{2} + \frac{\varepsilon}{2} = \varepsilon$ $\Rightarrow$
$\{x^{(n)}\}_{n=1}^{\infty}$ - фундаентальная

\vspace{0.5cm}

2. $\{x^{(n)}\}_{n=1}^{\infty}$ - фундаментальная 

Зададим $\varepsilon = 1$ $\exists N_1 \in \mathbb {N}$ : $\forall n, m > N_1$ $\Rightarrow$ $\rho (x^{(n)}, x^{(m)})  < 1$

Пусть $m = N_1$  $\forall n \ge N_1$ : $\rho (x^{(n)}, x^{(m)})  < 1$, т.е. $x^{(N_1)} - 1 < x^{(n)} < x^{(N_1)} + 1$ $|x^{(n)}| \leq |x^{(N_1)}| + 1$

$A =  |x^{(N_1)}| + 1$ $\Rightarrow$ $|x^{(n)}| \leq A \forall n \in \mathbb {N} $  $\Rightarrow$ $\{x^{(n)}\}$ - ограничена 

Так как фундаментальная последовательность $\{x^{(n)}\}$ является ограниченной, то по теореме Больцано-Вейерштрасса она содержит сходящуюся подпоследовательность $\{x^{({n_k})}\}$. Пусть ее предел равен $a$, т.е. $\lim_{k\to\infty} x^{(n_k)} = a$

$\forall \varepsilon > 0$ $\exists N_2 \in \mathbb {N}$ $\forall k > N_2$ $\Rightarrow$ $\rho (x^{(n_k)}, a)  < \frac{\varepsilon}{2}$

$N_\varepsilon = \max(N_1, N_2)$ 
$n_k \ge N_\varepsilon$
Тогда при $m = n_k \forall n > N_\varepsilon$ 

$\rho (x^{(n)}, x^{(n_k)}) < \frac{\varepsilon}{2}$

$\rho (x^{(n)}, a) \leq \rho (x^{(n)}, x^{(n_k)}) + \rho (x^{(n_k)}, a) < \frac{\varepsilon}{2} + \frac{\varepsilon}{2} = \varepsilon$, т.е.

$\lim_{n\to\infty} x^{(n)} = a$
