\section{Понятие условного экстремума функции многих переменных. 
Методы исследования на условный экстремум.}
\subsection{Определение условного экстремума.}
Говорят, что если $f(x_1, x_2, \dots, x_{n+m})$ - функция от $n+m
$ переменных, $m$ из которых подчинены уравнениям связи
$$\Phi_i(x_1,\dots, x_n, x_{n+1},\dots, x_{n+m})$$
$$(i = 1, 2,\dots, m)$$то в точке $M_0(x_1^0, x_2^0,
\dots, x^0_{n+m})$, удовлетворяющей уравнениям связи, $f$ имеет условный максимум(минимум), если неравенство:
\begin{align*} f(x_1, \dots, x_{n+m} &\leq f(x_1^0, \dots, x^0_{n+m})\\
	&(\geq)
\end{align*}
выполняется в некоторой окрестности точки $M_0$, чьи все точки удовлетворяют уравнениям связи.\\

Условным экстремумом называются условный максимум, минимум.
\subsection{Методы исследования на условный экстремум.}
\subsubsection{Необходимые условия.}
Будем исходить из предположения, что и $f$, и $\Phi_i$ имеют
в окрестности рассматриваемой точки непрерывные частные производные
по всем аргументам. Далее, пусть в точке $M_0$ отличен от нуля хотя бы один из определителей $m$-ого порядка, составленных из матрицы частных производных (матрица имеет в точке ранг $m$):
$$\begin{matrix}
	{\partial\Phi_1 \over \partial x_1}& \dots &
	{\partial\Phi_1 \over \partial x_n}&
	{\partial\Phi_1 \over \partial x_{n+1}}&
	\dots&
	{\partial\Phi_1 \over \partial x_{n+m}}\\
	{\partial\Phi_2 \over \partial x_1}& \dots &
	{\partial\Phi_2 \over \partial x_n}&
	{\partial\Phi_2 \over \partial x_{n+1}}&
	\dots&
	{\partial\Phi_2 \over \partial x_{n+m}}\\
	\hdotsfor{6}\\
	{\partial\Phi_m \over \partial x_1}& \dots &
	{\partial\Phi_m \over \partial x_n}&
	{\partial\Phi_m \over \partial x_{n+1}}&
	\dots&
	{\partial\Phi_m \over \partial x_{n+m}}\\
\end{matrix}$$
Тогда, если ограничится достаточно малой окрестностью точки $M_0$, исходная система равносильна системе вида:
$$x_{n+1} = \phi(x_1,\dotsc, x_n),\dotsc, x_{n+m} = \phi_m(x_1,\dotsc,x_n)$$
где $\phi_1,\dotsc, \phi_m$ - неявные функции, определяемые исходной системой. Требование, что $x_{n_1}, \dotsc, x_{n+m}$ удовлетворяют уравнениям связи, можно заменить предположением, что эти переменные представляют собой функции от $x_1,\dotsc, x_n$.\\

Таким образом, вопрос об условных экстремумах функции с $n+m$ переменными можно перевести в вопрос об абсолютных экстремумах сложной функции с $n$ переменными.\\
\\

Метод неопределённых множителей Лагранжа.\\

Метод неопределённых множителей Лагранжа предлагает более удобную альтернативу исследованию условных экстремумов, не смешивая зависимые и независимые переменные.\\

Продифференцируем уравнения
$$\sum_{j = 1}^{n+m} {\partial \Phi_i \over \partial x_j}dx_j = 0 
\quad (i = 1,2,\dotsc,m)$$
И домножим равенства на произвольные множители $\lambda_i (i = 1,2,\dotsc,m)$
$$\sum_{j=1}^{n+m}\left(
{\partial f \over \partial x_j} + 
\lambda_1{\partial \Phi_1 \over \partial x_j} +
\dotsb +
\lambda_m{\partial \Phi_m \over \partial x_j} \right)dx_j
=0
$$
где $dx_{n+1}, \dotsc, dx_{n+m}$ - дифференциалы неявных функций; производные вычислены в точке $M_0$\\

Выберем теперь значения множителей $\lambda_i = \lambda_i^0 (i = 1, \dotsc, m)$ таким образом, чтобы обращались в нуль коэффициенты при зависимых дифференциалах.

$$
{\partial f \over \partial x_j} + 
\lambda_1^0{\partial \Phi_1 \over \partial x_j} +
\dotsb +
\lambda_m^0{\partial \Phi_m \over \partial x_j} = 0
$$
$$j=n+1,\dotsc,n+m$$
При выбранных значениях равенство принимает вид
$$\sum_{j=1}^{n}\left(
{\partial f \over \partial x_j} + 
\lambda_1^0{\partial \Phi_1 \over \partial x_j} +
\dotsb +
\lambda_m^0{\partial \Phi_m \over \partial x_j} \right)dx_j
= 0
$$
Здесь мы имеем дело с дифференциалами независимых переменных, поэтому коэффициенты
при них должны быть нулями. Накладывая новый ограничения на систему, имеем:
$$
{\partial f \over \partial x_j} +
\lambda_1^0{\partial \Phi_1 \over \partial x_j} +
\dotsb +
\lambda_m^0{\partial \Phi_m \over \partial x_j} = 0
$$
$$(j = 1, 2, \dotsc, n)$$
Итак, для определения $n+m$ неизвестных $x_1, \dotsc, x_{n+m}$, а так же $m$ множителей $\lambda_1,\dotsc, \lambda_m$, имеем $m$ уравнений связи, и $n+m$ уравнений
$$
{\partial f \over \partial x_j} +
\lambda_1{\partial \Phi_1 \over \partial x_j} +
\dotsb +
\lambda_m{\partial \Phi_m \over \partial x_j} = 0
$$
$$(j = 1, 2, \dotsc n+m)$$

Часто для облегчения жизни вводят вспомогательную функцию
$$F = f + 
\lambda_1\Phi_1 + 
\dotsb + 
\lambda_m\Phi_m
$$
Тогда уравнения записываются как
$${\partial F \over \partial x_j} = 0 \quad (j = 1, 2, \dotsc, n+m)$$

Если полученная система имеет решение относительно $x_i$ и $\lambda_k$, то $M_0$ может быть точкой условного экстремума: метод Лагранжа носит необходимый, но не достаточный характер. Далее, существенную роль играет предположение о ранге матрицы; при решении задач бы неплохо - для уверенности - проверять, что предположение выполняется.
\subsubsection{Достаточные условия.}
Пусть существуют и непрерывны вторые производные для функций $f$ и $\Phi_j
(j = 1, 2,\dotsc, m)$. Пусть точка $M_0(x_1^0, \dotsc, x_{n+m}^0)$, а так же множители $\lambda_1^0,\dotsc,\lambda_0^m$ удовлетворяют необходимым условиям.\\

По определению, вопрос о наличии в точке условного максимума (минимума) зависит, как и вопрос об абсолютном экстремуме, от разности
$$\Delta = f(x_1,\dotsc, x_{n+m}) - f(x_1^0, \dotsc, x^0_{n+m})$$
с той лишь разницей, что точка $(x_1,\dotsc, x_{n+m})$ должна удовлетворять уравнениям связи - или равносильной им системе.\\

Разность $f$ можно заменить на $F$, если $\lambda_i = \lambda_i^0$. Запишем это приращение по формуле Тейлора:

$$\Delta = {1 \over 2}\left\{
	\sum_{j,k = 1}^{n+m}
	A_{jk}
	\Delta x_j
	\Delta x_k
	+
	\sum_{j,k = 1}^{n+m}
	\alpha_{jk}
	\Delta x_j
	\Delta x_k
	\right\}$$
	где
	$$\Delta x_j = x_j - x_j^0, \quad A_{jk} = F_{x_j, x_k}^{''}(x_1^0,
	\dotsc, x^0_{n+m})$$
	$$(j, k = 1, 2, \dotsc, n+m)$$
	и $\alpha_{jk} \to 0$ если $\Delta x_1 \to 0, \dotsc, \Delta x_n \to 0$\\

	Если заменит все приращения $\Delta x_j$ на $dx_j$ то ничего, что касается независимых переменных, не изменится, а что касается зависимых переменных,
	то нам лишь придётся заменить одну б.м. $\alpha_{jk}$ на другую.\\

	Ловким движением руки мы получили квадратичную форму.\\

	Аналогично случаю абсолютного экстремума доказывается, что если форма определенная и положительна (отрицательна), то точка является локальным минимумом (максимумом).
