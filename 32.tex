\section{Зависимость (независимость) функций.}

\subsection{Понятие линейной зависимости (независимости) функций}

\begin{definition}
Функции $y_1(x)$, $y_2(x)$, ..., $y_k(x)$ называются линейно зависимыми на интервале $(a, b)$, если существуют постоянные $c_1$, $c_2$, ..., $c_k$ не равные нулю одновременно и такие, что 
\begin{equation}
    c_1y_1(x) + c_2y_2(x) + ... + c_ky_k(x) = 0, x \in (a, b)
\end{equation}
    
В противном случае функции $y_1(x)$, $y_2(x)$, ..., $y_k(x)$ называются линейно независимыми на интервале $(a, b)$.
\end{definition}

\begin{definition}
Вектор-функции $y^1(x)$, ..., $y^k(x)$ называются линейно зависимыми на интервале $(a, b)$, если существуют постоянные $c_1$, ..., $c_k$ не равные нулю одновременно и такие, что

\begin{equation}
    c_1y^1(x) + c_2y^2(x) + ... + c_ky^k(x) = 0, x \in (a, b)
\end{equation}

В противном случае эти вектор-функции $y_1(x)$, $y_2(x)$, ..., $y_k(x)$ называются линейно независимыми на интервале  $(a, b)$.
\end{definition}

\subsection{Определитель Вронского}

\begin{definition}
Пусть $y^1(x)$, ..., $y^k(x)$ - вектор-функции с n компонентами. Определитель 

\begin{equation}
    w(x) = det(y^1(x), y^2(x), ..., y^k(x))
\end{equation}

называется определителем Вронского набора функции {$y^1(x)$, ..., $y^k(x)$}.
\end{definition}

Далее предполагаем, что $x \in I = (a, b)$, все рассматриваемые вектор-функции непрерывны, и их линейная зависимость исследуется при $x\in I$.

\begin{theorem}. 
    Если определитель Вронского системы вектор-функции $y^1(x)$, ..., $y^k(x)$ отличен от нуля хотя бы в одной точке $x_0\in I$, то эти вектор-функции линейно независимы.
\end{theorem}

\begin{proof}:
Допустим,  что вектор-функции линейно зависимы, тогда сущетсвуют постоянные $c_1$, ..., $c_k$ не равные нулю и такие, что 

\begin{equation}
    c_1y^1(x) + c_2y^2(x) + ... + c_ny^n(x) = 0, x \in I
\end{equation}

В частности,

\begin{equation}
    c_1y^1(x_0) + c_2y^2(x_0) + ... + x_ky^k(x_0) = 0
\end{equation}

Так как $w(x_0)!=0$, то векторы $y^1(x_0)$, ..., $y^x(x_0)$ линейно независимы и потому все постоянные $c_k$ равны нулю.
\end{proof}

\begin{theorem}
Если веткор-функции $y^1(x)$, ..., $y^n(x)$ линейно зависимы, то их определитель вронского тождественно равен нулю.
\end{theorem}

\begin{proof}
следует из того, что если столбцы определителя линейно зависимы, то определитель равен нулю. 
\end{proof}

Рассмотрим однородную линейную систему из n уравнений 

\begin{equation}
    \frac{dy}{dx} = A(x)y
\end{equation}

с непрерывной при $x\in I$ матрицей-функцией $A(x)$.

\begin{theorem}.
Пусть вектор-функции $y^1(x)$, ..., $y^n(x)$ - решение системы (6). Если их определитель Вронского $w(x)$ обращается в нуль, хотя бы в одной точке $x_0\in I$, то эти вектор-функции линейно зависимы.
\end{theorem}

\begin{proof}
Так как $w(x_0)=0$, то существуют постоянные $c_1$, ..., $c_n$, не равные нулю одновременно и таке, что выполняется тождество (5). Рассмотрим вектор-функцию

\begin{equation}
    y(x) = c_1y^1(x) + ... + c_ny^n(x)
\end{equation}

которая есть решение системы (6) и $y(x_0) = 0$, в силу (5). Вектор-функция $\widetilde{y}(x) = 0$ удовлетворяет системе (7) и имеет те же данные Коши, что и $\widetilde{\widetilde{y}}(x)$, так как $\widetilde{y}(x_0) = 0$. По теореме единственности $y(x)=\widetilde{y}(x)$, так что $y(x) = 0$, $x \in I$, и из (7) следует линейная зависимость вектор-функций $y^1(x)$, ..., $y^n(x)$.

\end{proof}

\begin{theorem}
Для произвольные вектор-функций утверждения леммы 3 неверно
\end{theorem}