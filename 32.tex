\documentclass{article}
 
%Russian-specific packages
%--------------------------------------
\usepackage[russian]{babel}
%--------------------------------------
 
%Hyphenation rules
%--------------------------------------
\usepackage{hyphenat}
%--------------------------------------

%Math
%--------------------------------------
\usepackage{mathtools}
\usepackage{mathrsfs}
\usepackage{amsmath}
\usepackage{amsfonts}
%-------------------------------------- 

\begin{document}

    \section{Зависимость (независимость) функций.}
    
    Функции $y_1(x)$, $y_2(x)$, ..., $y_k(x)$ называются линейно зависимыми на интервале $(a, b)$, если существуют постоянные $c_1$, $c_2$, ..., $c_k$ не равные нулю одновременно и такие, что 
    \begin{equation}
        c_1y_1(x) + c_2y_2(x) + ... + c_ky_k(x) = 0, x \in (a, b)
    \end{equation}
    
    В противном случае функции $y_1(x)$, $y_2(x)$, ..., $y_k(x)$ называются линейно независимыми на интервале $(a, b)$.
    
    Вектор функции $y^1(x)$, ..., $y^k(x)$ называются линейно зависимыми на интервале $(a, b)$, если существуют постоянные $c_1$, ..., $c_k$ не равные нулю одновременно и такие, что
    
    \begin{equation}
        c_1y^1(x) + c_2y^2(x) + ... + c_ky^k(x) = 0, x \in (a, b)
    \end{equation}
    
    В противном случае эти вектор-функции $y_1(x)$, $y_2(x)$, ..., $y_k(x)$ называются линейно независимыми на интервале  $(a, b)$.
    
    Пусть $y^1(x)$, ..., $y^k(x)$ - вектор-функции с n компонентами. Определитель 
    
    \begin{equation}
        w(x) = det(y^1(x), y^2(x), ..., y^k(x))
    \end{equation}
    
    называется определителем Вронского набора функции {$y^1(x)$, ..., $y^k(x)$}.
    
    Далее предполагаем, что $x \in I = (a, b)$, все рассматриваемые вектор-функции непрерывны, и их линейная зависимость исследуется при $x\in I$.
    
    \textbf{Лемма 1}. Если определитель Вронского системы вектор-функции $y^1(x)$, ..., $y^k(x)$ отличен от нуля хотя бы в одной точке $x_0\in I$, то эти вектор-функции линейно независимы.
    
    \textbf{Доказательство}:
    Допустим,  что вектор-функции линейно зависимы, тогда сущетсвуют постоянные $c_1$, ..., $c_k$ не равные нулю и такие, что 
    
    \begin{equation}
        c_1y^1(x) + c_2y^2(x) + ... + c_ny^n(x) = 0, x \in I
    \end{equation}
    
    В частности,
    
    \begin{equation}
        c_1y^1(x_0) + c_2y^2(x_0) + ... + x_ky^k(x_0) = 0
    \end{equation}
    
    Так как $w(x_0)!=0$, то векторы $y^1(x_0)$, ..., $y^x(x_0)$ линейно независимы и потому все постоянные $c_k$ равны нулю.
    
    \textbf{Лемма 2}.
    Если веткор-функции $y^1(x)$, ..., $y^n(x)$ линейно зависимы, то их определитель вронского тождественно равен нулю.
    
    \textbf{Доказательство:}
    следует из того, что если столбцы определителя линейно зависимы, то определитель равен нулю. 
    
    Рассмотрим однородную линейную систему из n уравнений 
    
    \begin{equation}
        \frac{dy}{dx} = A(x)y
    \end{equation}
    
    с непрерывной при $x\in I$ матрицей-функцией $A(x)$.
    
    \textbf{Лемма 3}.
    Пусть вектор-функции $y^1(x)$, ..., $y^n(x)$ - решение системы (6). Если их определитель Вронского $w(x)$ обращается в нуль, хотя бы в одной точке $x_0\in I$, то эти вектор-функции линейно зависимы.
    
    \textbf{Доказательство:}
    Так как $w(x_0)=0$, то существуют постоянные $c_1$, ..., $c_n$, не равные нулю одновременно и таке, что выполняется тождество (5). Рассмотрим вектор-функцию
    
    \begin{equation}
        y(x) = c_1y^1(x) + ... + c_ny^n(x)
    \end{equation}
    
    которая есть решение системы (6) и $y(x_0) = 0$, в силу (5). Вектор-функция $\tilde y(x) = 0$ удовлетворяет системе (7) и имеет те же данные Коши, что и $\tilde \tilde y(x)$, так как $\tilde y(x_0) = 0$. По теореме единственности $y(x)=\tilde y(x)$, так что $y(x) = 0$, $x \in I$, и из (7) следует линейная зависимость вектор-функций $y^1(x)$, ..., $y^n(x)$.
    
    \textbf{Замечание:}
    Для произвольные вектор-функций утверждения леммы 3 неверно
    
\end{document}