\section{Теорема Больцано-Вейерштрасса}

\subsection{Теорема 1} Из любой ограниченной последовательности векторов $m$-мерного пространства можно выделить сходящуюся подпоследовательность.

Пусть $\{x^{(n)}\}_{n=1}^{\infty}$ - ограничена в $\mathbb {R}^m$. Тогда $\exists \{x^{(n_k)}\}_{n=1}^{\infty} \subseteq \{x^{(n)}\}_{n=1}^{\infty}$: $\exists a \in \mathbb {R}^m$

$a = \lim_{k\to\infty} x^{(n_k)}$

\textcolor{red}{Доказательство}:

$x^{(n)} = (x^{(n)}_1, x^{(n)}_2, ..., x^{(n)}_m)$ - ограничена, т.е. $ x^{(n)} \in B(0, R) \Rightarrow$ 

$ \forall i=(1, 2,..., m) \{x^{(n)}_i\} $ (коорд. послед.) - ограничена в $\mathbb {R}$, т.к. $ |x_i|\leq \rho(x, 0) $  (например, $ \rho_ 1 (x, 0) |x_i| \leq \sum_{i=1}^{m}|x_i|$)
\vspace{1cm}

$\{x^{(n)}_1\}^{\infty}_{n =1}$ - огр. в $\mathbb {R}$ $\Rightarrow $ $\exists \{x^{(n_{k_1})}_1\}_{k_1=1}^{\infty} \subseteq \{x^{(n)}_1\}_{n=1}^{\infty}$ : $ \exists \lim_{k_1\to\infty} x_1^{(n_{k_1})} = a_1 \in \mathbb {R}$ (по лемме Больцано-Вейерштрасса для последовательности)
\vspace{1cm}

Теперь рассмотрим:

$ x^{(n_{k_1})} = (x^{(n_{k_1})}_1 , x^{(n_{k_1})}_2 , ..., x^{(n_{k_1})}_m)$, где $x^{(n_{k_1})}_1 \rightarrow a_1$, при $k_1 \rightarrow \infty$

Выделим сход. подпоследовательность для $\{x^{(n_{k_1})}_2\}$ - ограничена в $\mathbb {R}$ $\Rightarrow $
$\exists \{x^{(n_{{k_1}_{k_2}})}_2\}_{k_2=1}^{\infty} \subseteq \{x^{(n_{k_1})}_2\}_{k_1=1}^{\infty}$: $ \exists \lim_{k_2\to\infty} x^{(n_{{k_1}_{k_2}})} = a_2 \in \mathbb {R}$
\vspace{1cm}

$ x^{(n_{{k_1}_{k_2}})} = (x^{(n_{{k_1}_{k_2}})}_1, x^{(n_{{k_1}_{k_2}})}_2 , ..., x^{(n_{{k_1}_{k_2}})}_m)$, где

$x^{(n_{{k_1}_{k_2}})}_1 \rightarrow a_1$, при $k_2 \rightarrow \infty$ (по свойству о сходимости подпосл.)

$x^{(n_{{k_1}_{k_2}})}_2 \rightarrow a_2$, при  $k_2 \rightarrow \infty$

$...$

на $m$ шаге:

$x^{(n_{{{k_1}_{{...}_{k_m}}}})} = (x^{(n_{{{k_1}_{{...}_{k_m}}}})}_1, x^{(n_{{{k_1}_{{...}_{k_m}}}})}_2, ..., x^{(n_{{{k_1}_{{...}_{k_m}}}})}_m)$, где 

$x^{(n_{{{k_1}_{{...}_{k_m}}}})}_1 \rightarrow a_1$, 
$x^{(n_{{{k_1}_{{...}_{k_m}}}})}_2 \rightarrow a_2$,
$...$,

$x^{(n_{{{k_1}_{{...}_{k_m}}}})}_m \rightarrow a_m$, при  $k_m \rightarrow \infty$
\vspace{1cm}

По критерию сходимости в $\mathbb {R}^m$:

$ x^{(n)}_i \rightarrow a_i \Rightarrow x^{(n)} \rightarrow a$
\vspace{1cm}

$\{x^{(n)}\}_{n=1}^{\infty}$ - \textcolor{red}{бесконечно большая}, если $\rho(x^{(n)}, 0) \rightarrow \infty$, при $n \rightarrow \infty$
т.е. $\forall \epsilon > 0$ $\exists n_\epsilon \in \mathbb {N}$ $\forall n>n_\epsilon$ $\rho(x^{(n)}, 0) = |x^{(n)}|>\epsilon$
\vspace{1cm}

\subsection{Теорема 2} Из любой неограниченной последовательности векторов $m$-мерного пространства можно выделить бесконечно большую подпоследовательность.

\textcolor{red}{Доказательство}:

$\{x^{(n)}\}_{n=1}^{\infty}$ - неограничено $\Rightarrow \exists k = 1,..,m$: $\{x_k^{(n)}\}_{n=1}^{\infty}$ - неограничено (иначе, если все координатные последовательности будут ограничены, то по критерию сходимости $\{x^{(n)}\}_{n=1}^{\infty}$ будет ограничена)

Пусть $x_k^{(n_{{p}})} \rightarrow \infty$, при $p \rightarrow \infty$ 

$\{x_k^{(n_p)}\}_{p=1}^{\infty} \subseteq \{x_k^{(n)}\}_{n=1}^{\infty}$

Tогда последовательность векторов будет б.б.:

$ \rho(x^{(n_p)}, 0) \ge |x^{(n_p)}_k| \rightarrow \infty$
