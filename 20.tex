\documentclass[a4paper,14pt]{article}
 
%Russian-specific packages
%--------------------------------------
\usepackage{cmap}					
\usepackage[english,russian]{babel}
%--------------------------------------
 
%Hyphenation rules
%--------------------------------------
\usepackage{hyphenat}
%--------------------------------------

%Math
%--------------------------------------
\usepackage{mathtools}
\usepackage{mathrsfs}
\usepackage{amsmath}
\usepackage{amsfonts}
\usepackage{amsmath,amsfonts,amssymb,amsthm,mathtools}
%-------------------------------------- 

%Hyperlinks
%--------------------------------------
\usepackage[svgnames]{xcolor}
\usepackage[colorlinks=true, linkcolor=Blue, urlcolor=Blue]{hyperref}
%--------------------------------------
\usepackage[14pt]{extsizes}

\linespread{1.15}

\pagestyle{empty}

\newtheorem{theorem}{Теорема}
\newtheorem{definition}{Определение}

\begin{document}

    \section{Необходимые условия дифференцируемости. Достаточные условия дифференцируемости.}

    Рассмотрим полное приращение функции $f(x,y)$ в точке $(x_0,y_0)$:

    $$\Delta{f(x_0,y_0)}=f(x_0+\Delta{x},y_0+\Delta{y})-f(x_0,y_0)$$

    \begin{definition}
        Функция $f(x,y)$ называется \textit{дифференцируемой} в точке $(x_0,y_0)$, если существуют числа $A$ и $B$
        такие, что её полное приращение в этой точке представимо в виде
    \end{definition}

    $$\Delta{f(x_0,y_0)}=A\Delta{x}+B\Delta{y}+o(\rho),$$

    где $\rho=\sqrt{\Delta{x^2}+\Delta{y^2}}\to{0}$ при $(\Delta{x},\Delta{y})\to(0,0)$.

    \begin{theorem}
        Если Функция $f(x,y)$ дифференцируема в точке $(x_0,y_0)$, то она имеет в этой точке частные производные по каждому аргументу $x,y$.
        При этом $\frac{\delta{f}}{\delta{x}}(x_0,y_0)=A$, $\frac{\delta{f}}{\delta{y}}(x_0,y_0)=B$,
        где $A$ и $B$ - числа из равенства выше.
    \end{theorem}

    \textit{Доказательство.}
    В определении дифференцируемости положим $\Delta{y}=0$, a $\Delta{x}\neq{0}$ и получим

    $$f(x_0+\Delta{x},y_0)-f(x_0,y_0)=A\Delta{x}+o(\Delta{x}).$$

    Поделим на $\Delta{x}$ и посчитаем предел

    $$\lim_{\Delta{x}\to{0}}\frac{f(x_0+\Delta{x},y_0)-f(x_0,y_0)}{\Delta{x}}=\lim_{\Delta{x}\to{0}}A+\frac{o(\Delta{x})}{\Delta{x}}=A,$$

    т. е. существует $\frac{\delta{f}}{\delta{x}}(x_0,y_0)$  она равна $A$.

    Учитывая доказанное свойство, условие дифференцируемости функции можно записать в виде

    $$\Delta{f(x_0,y_0)}=\frac{\delta{f}}{\delta{x}}(x_0,y_0)\Delta{x}+\frac{\delta{f}}{\delta{y}}(x_0,y_0)\Delta{y}+o(\sqrt{\Delta{x^2}+\Delta{y^2}}).$$

    Часто бывает удобна следующая (эквивалентная) форма определения дифференцируемости.

    \begin{definition}
        Функция $f(x,y)$ называется \textit{дифференцируемой} в точке $(x_0,y_0)$, если её полное приращение в этой точке представимо в виде
    \end{definition}

    $$\Delta{f(x_0,y_0)}=\frac{\delta{f}}{\delta{x}}(x_0,y_0)\Delta{x}+\frac{\delta{f}}{\delta{y}}(x_0,y_0)\Delta{y}+\alpha\cdot\Delta{x}+\beta\cdot\Delta{y},$$
    
    где функции $\alpha=\alpha(\Delta{x},\Delta{y})$ и $\beta=\beta(\Delta{x},\Delta{y})$ - бесконечно малые при $(\Delta{x},\Delta{y})\to(0,0)$.
\end{document}