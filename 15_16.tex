\newcommand\RRR{\ensuremath{\mathbb{R}}}
\newcommand{\rom}[1]
    {\MakeUppercase{\romannumeral #1}}
\newtheorem{theorem}{Теорема}[section]
\theoremstyle{definition}
\newtheorem{definition}{Определение}[section]
\newpage
\section{Теоремы Вейерштрасса}
    
\begin{theorem}[\rom{1} теорема Вейерштрасса]
    Функция, непрерывная на компакте(замкнутое ограниченное множество), ограничена.
\end{theorem}

\begin{proof}
    $f$ определена на компакте $D \subset \RRR^m$. Проведем доказательство для числового случая, затем обобщим на $m$-мерное пространство.
    \[
        f:\; D \rightarrow \RRR^1
    \]
    О/п: $f$ - неограничена: $\forall C > 0 \quad \exists x \in D:\;|f(x)| > C$. 
    Построим последовательность, будем брать разные $C$:
    \[
        C=n \quad \exists x^{(n)} \in D:\; |f(x^{(n)})| > n
    \]
    Из любой ограниченой последовательности можно выделить сходящуюся подпоследовательность:
    \[
        \{x^{(n)}\} \subset D\text{ - ограничено }\Rightarrow \exists \{ x^{(n_k)} \} \subseteq \{x^{(n)}\}:\; x^{(n_k)} \xrightarrow[k \rightarrow \infty]{} a
    \]
    Т.к. $D$ - замкнутое(содержит свои предельные точки), то $a \in D$. Тогда
    \[
        f \text{ непрерывна в } a \Rightarrow f(x^{(n_k)}) \xrightarrow[k \rightarrow \infty]{} f(a) \Rightarrow \{f(x^{(n_k)})\} \text{ - ограничена} 
    \]
    По построению 
    \[
        |f(x^{(n_k)})| > n_k  \xrightarrow[k \rightarrow \infty]{} \infty
    \]
    Т.е. подпоследовательность $\{f(x^{(n_k)})\}$ - б.б. Получили противоречие. 
        
    Рассмотрим теперь не числовую функцию, а вектор-функцию.
    Док-во аналогично, только модуль меняется на метрику.

    О/п: $f$ - неограничена:
    \begin{gather*}
        \forall C > 0 \quad \exists x \in D:\;|f(x)| = \rho(f(x), 0) > C\\
        C=n \quad \exists x^{(n)} \in D:\; |f(x^{(n)})| = \rho(f(x^{(n)}), 0) > n\\
        \{x^{(n)}\} \subset D\text{ - ограничено }\Rightarrow \exists \{ x^{(n_k)} \} \subseteq \{x^{(n)}\}:\; x^{(n_k)} \xrightarrow[k \rightarrow \infty]{} a \in D\\
        \Rightarrow f(x^{(n_k)}) \xrightarrow[k \rightarrow \infty]{\RRR^p} f(a)\\
        |f(x^{(n_k)})| = \rho(f(x^{(n_k)}), 0) > n_k  \xrightarrow[k \rightarrow \infty]{} \infty
    \end{gather*}

\end{proof}

\newpage
\begin{theorem}[\rom{2} теорема Вейерштрасса]
    Функция $f:\; D \subseteq \RRR^m \rightarrow \RRR$(только для одномерного случая, т.к нет понятия точных граней для вектор-функции), 
    непрерывная на компакте(замкнутое ограниченное множество), достигает своих точных граней, т.е
    \[\exists a, b \in D:\; f(a)=\sup_{x\in D}{f(x)} =M , f(b)=\inf_{x\in D}{f(x)}=m \]
\end{theorem}

\begin{proof}
    Функция ограничена(по первой теореме) $\Rightarrow$ имеет точную верхную грань: 
    \[\exists M = \sup_{x\in D}{f(x)}\]
    \begin{enumerate}
        \item $\forall x \in D\; f(x) \leqslant M$
        \item $\forall \varepsilon > 0\; \exists x^{(\varepsilon)} \in D:\; f(\varepsilon) > M - \varepsilon$
    \end{enumerate}
    Построим последовательность по 2 пункту:
    \[
        \varepsilon_n > 0, \varepsilon_n \rightarrow 0 \quad \exists x^{(n)} \in D:\; M-\varepsilon_n < f(x^{(n)}) \leqslant M
    \]
    Последовательность лежит в ограниченом множестве $\Rightarrow$ она ограничена $\Rightarrow$ 
        существует сходящаяся подпоследовательность, выделим ее. 
    \[
        \exists \{x^{(n_k)}\} \subseteq \{x^{(n)}\}:\; x^{(n_k)} \xrightarrow[k \rightarrow \infty]{} a
    \]
    Т.к. $D$ - замкнутое(содержит свои предельные точки), то $a \in D$(как в предыдущей теореме). Тогда
    \[
        f \text{ непрерывна в } a \Rightarrow f(x^{(n_k)}) \xrightarrow[k \rightarrow \infty]{} f(a)
    \]
    С другой стороны:
    \begin{gather*}
        \varepsilon_{n_k} > 0, \varepsilon_{n_k} \rightarrow 0 \quad \exists x^{(n_k)} \in D:\; M-\varepsilon_{n_k} < f(x^{(n_k)}) \leqslant M\\
        M-\varepsilon_{n_k} \xrightarrow[n \rightarrow \infty]{} M\\
        M \rightarrow M\\
        f(x^{(n_k)}) \rightarrow M
    \end{gather*}
    В силу единственности предела $f(a) = M =\sup{f(x)}$. Аналогично для $\inf$.
\end{proof}

\newpage
\section{Равномерная непрерывность. Теорема Кантора}

\begin{definition}
    $f:\; D \subseteq \RRR^m \rightarrow \RRR^k$. 

    $f$ равномерна непрерывна на $D$, если
    \[
        \forall \varepsilon > 0\;\; \exists \delta(\varepsilon) > 0 \quad
        \forall x, y \in D \quad \rho_m(x, y) < \delta(\varepsilon) \Rightarrow \rho_k(f(x), f(y)) < \varepsilon
    \]
\end{definition}

\begin{theorem}[Теорема Кантора]
    Если $f$ непрерывна на компакте $D$, то она равномерно непрерывна на $D$.
\end{theorem}

\begin{proof}
    Аналогично числовому случаю.

    О/п:
    \[
        \exists \varepsilon > 0\;\; \forall \delta > 0 \quad
        \exists x, y \in D \quad \rho_m(x, y) < \delta \wedge \rho_k(f(x), f(y)) \geqslant \varepsilon
    \]
    Поскольку для $\delta$ мы можем брать разные велечины:
    \[
        0 < \delta_n  \xrightarrow[n \rightarrow \infty]{} 0 \quad
        \exists x^{(n)}, y^{(n)} \in D \quad \rho_m(x^{(n)}, y^{(n)}) < \delta_n \Rightarrow \rho_k(f(x^{(n)}), f(y^{(n)})) \geqslant \varepsilon
    \]
    $\{x^{(n)}\}$ ограниченная, можно выделить сходящуюся подпоследовательность. Ее предел лежит в множестве, т.к. оно замкнутое:
    \[
        \exists \{x^{(n_p)}\} \in \{x^{(n)}\}: \quad x^{n_p} \xrightarrow[p \wedge \infty]{} a \in D
    \]
    Рассмотрим подпоследовательность $\{y^{(n_p)}\}$ 
    и заметим, что она будет сходиться так же к $a$, так как по построению
    $\rho_m(x^{(n_p)}, y^{(n_p)}) < \delta_{n_p}$, т.е. расстояние стремится к 0, образы стремятся к одной точке.
    В силу непрерывности функции:
    \begin{gather*}
        f(x^{(n_p)}) \rightarrow f(a)\\
        f(y^{(n_p)}) \rightarrow f(a)
    \end{gather*}    
    Тогда $\rho(f(x^{(n_p)}), f(y^{(n_p)})) \rightarrow 0$, но по построению $\rho(f(x^{(n)}), f(y^{(n)})) \geqslant \varepsilon$. Получили противоречие.
\end{proof}
